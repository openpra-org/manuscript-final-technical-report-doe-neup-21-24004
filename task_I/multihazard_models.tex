\subsection{Backgrounder on Multi-Hazard PRA}

Multi-hazard probabilistic risk assessment (PRA) addresses the challenge of analyzing multiple potential threats to a nuclear power plant (NPP) when hazards occur either concurrently or in a sequence. In general, these interactions can be complex because hazards may have non-trivial correlations or can trigger one another (e.g., a seismic event inducing a landslide). For instance, Choi et al.~\cite{Choi2021review} note that multi-hazard scenarios encompass both simultaneous and successive occurrences of distinct hazards that together can compromise plant safety. Although several approaches to multi-hazard analysis exist, no single framework is commonly adopted across the nuclear industry, resulting in varied levels of rigor and completeness.

The 2011 Fukushima Daiichi accident exemplified how multiple hazard events can combine to overwhelm existing defense-in-depth measures. According to the World Health Organization (WHO)~\cite{Great}, a magnitude 9.0 earthquake struck off Japan's eastern coast, followed by a tsunami that reached the NPP site. The International Nuclear and Radiological Event Scale (INES) rated this incident as a Level~7 accident~\cite{International}, highlighting the severity of multi-hazard effects. Prior to Fukushima, the Three Mile Island accident had already underscored the value of PRA for internal events. Fukushima, however, demonstrated the pressing need for extended methodologies that systematically capture interactions among external hazards (e.g., earthquakes, floods) and any ensuing internal initiators (e.g., loss-of-coolant accidents, equipment failures).

Expanding to a multi-hazard perspective involves more than simply combining single-hazard analyses. Wang et al.~\cite{Wang2020review} discuss how correlation can invalidate assumptions of hazard independence, requiring methods that account for conditional probabilities or coupling parameters. In a similar vein, Kappes et al.~\cite{Kappes2012Challenges} observe that each hazard type features unique physical processes (e.g., seismic ground motion vs. high-wind loading), necessitating hazard-specific models. Even so, frameworks must integrate these individual models in a coherent system analysis, since mischaracterizing inter-hazard relationships may lead to inaccurate risk estimates.

An operational step toward such integration is the categorization of hazards based on both the number of events and the order in which they arise. Kim et al.~\cite{Kim2019PRELIMINARY} emphasize that sequential occurrence (e.g., an earthquake followed by a secondary hazard like a fire) can induce different plant responses compared to purely simultaneous events. Table~1 (in the original text) illustrates one proposed classification scheme, highlighting that the temporal relationship of multi-hazard events is as critical as the physical severity of each.

In multi-hazard studies, the link between a hazard event and a PRA initiating event must be explicitly defined. Typically, an external hazard corresponds to a measurable intensity parameter (e.g., flood height, wind speed), and once this parameter reaches a threshold sufficient to challenge the plant, it becomes an initiating event. In more complex cases, a single hazard may induce a secondary one (e.g., an earthquake causing a landslide~\cite{Kwag2018Development}), or a hazard can exacerbate ongoing operational vulnerabilities. Yu et al.~\cite{Yu2015Large} provide an example in which a large-break loss-of-coolant accident (LBLOCA), normally categorized as an internal event, arises during a seismic disturbance.

Hazards are often screened or ranked prior to detailed analysis. Prosek et al.~\cite{Prosek2017Methodology} detail a methodology for selecting initiating events and external hazards for extended PRA. Similarly, Daniell et al.~\cite{Daniell2019Review} propose a structured progression from data collection and hazard identification to the final selection of those hazards needing in-depth evaluation. Although multiplying the frequencies of individual hazards can be computationally convenient, it may produce erroneous estimates if dependencies are overlooked. In fact, some multi-hazard assessments incorporate Bayesian networks to capture inter-hazard effects, as demonstrated by Kwag and Gupta~\cite{Kwag2017Probabilistic}.

In practice, multi-hazard PRA efforts remain a work in progress. International initiatives, such as the NARSIS project~\cite{Home}, strive to unify methodologies for characterizing hazard combinations and their potential impacts on NPP systems. Several studies--ranging from multi-hazard community-level assessments~\cite{Li2009Ranking} to site-wide Level~2 PRA strategies~\cite{Cooper2013What}--demonstrate that corporate and regulatory stakeholders are increasingly aware of the complexities involved. Even so, methodological gaps persist, particularly regarding systematic ways to quantify concurrent and successive hazards and translate them into PRA metrics. The industry consequently benefits from frameworks that (1) define hazard intensities and initiating events with precision, (2) incorporate hazard correlations or causal chains rigorously, and (3) integrate external-induced internal events in a unified probabilistic model.
Concurrent and successive occurrences of more than one hazard are defined as multi-hazard \cite{Choi2021review}. In the nuclear industry, multi-hazards are often overlooked in \acrshort{pra} since no general framework is available for such an analysis.

Multi-hazard \acrshort{pra} became a topic after the Fukushima \acrshort{npp} accident in March 2011. According to the \acrshort{who} \cite{Great}, the Great East Japan Earthquake was a 9.0 magnitude earthquake followed by a tsunami on the eastern coast. According to the \acrshort{ines} \cite{International}, this multi-hazard event led to a level~7 accident at the Fukushima Daiichi \acrshort{npp}, which is the highest severity level on the scale. While the Three Mile Island accident in the US highlighted the importance of \acrshort{pra}, the Fukushima Daiichi accident showed the necessity for multi-hazard \acrshort{pra}.

Multi-hazard \acrshort{pra} is inherently more complex, and deciding whether it is essential for a given facility is more challenging than single-hazard analysis \cite{Wang2020review}. As noted, significant nuclear accidents motivated the need for \acrshort{pra}, but additional factors such as climate change and population growth have amplified the frequency of local, regional, or global hazards. Such changes may lead to greater impacts on critical infrastructures, including \acrshort{npp}s, than originally anticipated at the design stage.

Unlike single hazard analysis, multi-hazard \acrshort{pra} requires different methods for different hazards, as each hazard has unique characteristics \cite{Kappes2012Challenges}. A feasible approach could be to categorize multi-hazards and define standard parameters to better understand and find practical ways to quantify multi-hazard \acrshort{pra}.

Although there are multiple ways to categorize multi-hazards, this work adopts the classification shown in Table~\ref{tab:MultiHazardCat} \cite{Kim2019PRELIMINARY}, which emphasizes how the order of events is just as important as the number of events when characterizing multi-hazards.

\begin{table}[htbp]
\centering
\caption{Multi-hazard categorization \cite{52}}
\label{tab:MultiHazardCat}
\begin{tabular}{p{0.3\textwidth} p{0.6\textwidth}}
\toprule
\textbf{Categories}                 & \textbf{Details}                                                             \\
\midrule
\multicolumn{2}{l}{\textit{Number of events and hazards}}  \\
Number of events & \textbf{Single event:} one event and one hazard\newline
\textbf{Multi-event:} two or more events, including a secondary event \\[6pt]
Number of hazards & \textbf{Single hazard:} one hazard (may be caused by a multi-event)\newline
\textbf{Multi-hazard:} two or more hazards (may be caused by a single event) \\[6pt]
\multicolumn{2}{l}{\textit{Order of event}} \\
Independent event [\textit{Independent}] & Two or more events that are independent of each other \\[6pt]
Simultaneous event [\textit{Concurrent}] & Two or more events caused by a single source \\[6pt]
Sequential event [\textit{Successive}] & Occurred by a secondary event \\
\bottomrule
\end{tabular}
\end{table}

% \begin{table}[htbp]
% \centering
% \caption{Multi-hazard categorization \cite{52}}
% \label{tab:MultiHazardCat}
% \begin{tabular}{p{0.3\textwidth} p{0.6\textwidth}}
% \toprule
% \textbf{Categories}                 & \textbf{Details}                                                             \
% \midrule
% \multicolumn{2}{l}{\textit{Number of events and hazards}}  \
% Number of events & \textbf{Single event:} one event and one hazard \newline
% \textbf{Multi-event:} two or more events, including a secondary event \[6pt]
% Number of hazards & \textbf{Single hazard:} one hazard (may be caused by a multi-event) \newline
% \textbf{Multi-hazard:} two or more hazards (may be caused by a single event) \[6pt]
% \multicolumn{2}{l}{\textit{Order of event}} \
% Independent event
% [\textit{Independent}]             & Two or more events that are independent of each other                       \[6pt]
% Simultaneous event
% [\textit{Concurrent}]              & Two or more events caused by a single source                                \[6pt]
% Sequential event
% [\textit{Successive}]              & Occurred by a secondary event                                               \ \bottomrule
% \end{tabular}
% \end{table}


Multiple definitions for multi-hazard \acrshort{pra} appear in the literature. Hence, providing descriptive definitions helps interpret Table~\ref{tab:MultiHazardCat}:

\begin{itemize}
\item \textbf{Hazards:} Phenomena that challenge the safe operation of an \acrshort{npp} (e.g., seismic or high wind).
\item \textbf{Hazard event:} An event caused by the occurrence of a specified hazard, described in terms of an intensity measure (e.g., peak ground acceleration for seismic hazards or wind speed for high wind hazards).
\item \textbf{Initiating events:} Natural or human-made perturbations to the plant that can challenge control and safety systems. Failure of these systems can lead to undesired consequences, such as a radioactive material release. An initiating event can result from internal (e.g., hardware fault, flood, fire) or external hazards (e.g., earthquakes, high winds).
\item \textbf{Hazard analysis:} Estimating the expected frequency of exceedance, over a specified time interval, of various levels of some characteristic measure of the hazard intensity (e.g., water level in a flood).
\item \textbf{Secondary hazard:} A hazard induced by another hazard (e.g., landslide caused by an earthquake).
\item \textbf{Multi-hazard:} A situation when one hazard occurs concurrently with another (e.g., seismic and flooding).
\item \textbf{Multi-hazard (initiating) event:} The occurrence of two or more correlated or uncorrelated events (e.g., an earthquake of a specific peak ground acceleration and high winds at a specific wind speed).
\end{itemize}

Although not all definitions are standard and further highlight the need for a common understanding in multi-hazard \acrshort{pra}, correctly accounting for the relationships between hazards, which can significantly complicate the analysis, is essential in any approach.

In multi-hazard \acrshort{pra}, both internal and external events must be considered. External events occur outside the \acrshort{npp}; their hazards may be natural (e.g., earthquakes, tsunamis) or man-made. However, multi-hazard \acrshort{pra} also addresses internal events when they are induced or exacerbated by external hazards. As an example, a large-break loss-of-coolant accident (LBLOCA) may occur during an earthquake~\cite{Yu2015Large}.

A widely recognized example is the Fukushima Daiichi \acrshort{npp} accident, an internal accident initially caused by an external earthquake and tsunami. A key lesson from this accident is that \acrshort{pra} for external events including combinations of events, and for external hazards capable of causing internal events must be revisited. In short, there is a pressing need for a framework that factors in (1) external events, (2) their combined likelihood and consequences, and (3) their interplay with internal events and plant-specific vulnerabilities.

Hazard events can occur independently or in combination. Two hazard events combined can occur simultaneously or within a short duration and may also trigger an internal event (e.g., equipment failure). This mirrors precisely the Fukushima Daiichi accident scenario. Disconnecting these events by assuming independence may be an oversimplification, as it can ignore correlations between hazards and lead to inadequate risk assessments.

Identifying individual hazards typically relies on screening analyses, plant design features, site characteristics, historical incident data, and hazard-impact experience \cite{Prosek2017Methodology}. In general, hazards can be addressed systematically in four steps \cite{Daniell2019Review}:

\begin{enumerate}
\item \textit{Initial data collection:} Site or plant-specific data are gathered, providing source material for subsequent analyses.
\item \textit{Identification of hazards:} Data are used to identify natural or man-made hazards relevant to the site.
\item \textit{Hazard screening:} Insignificant hazards or those with negligible effects on safety are screened out.
\item \textit{Detailed hazards analysis:} Relevant hazards that pass screening are analyzed in-depth to determine their impact on plant structures, systems, or components.
\end{enumerate}

A common practice has been to treat two or more hazard events as independent, multiplying their individual frequencies to approximate a combined event frequency. While this can be computationally convenient, it is not always justified.

Although multi-hazard \acrshort{pra} for nuclear applications remains an evolving field, several recent studies illustrate ongoing efforts. The NARSIS project \cite{Home} seeks to review, analyze, and improve safety assessment methodologies. One study \cite{Kwag2018Development} presented a practical approach for performing an earthquake-induced landslide \acrshort{pra} for \acrshort{npp}s. Another effort \cite{Kwag2017Probabilistic} surveyed multi-hazard \acrshort{pra} methods using a \acrshort{bn} approach and Bayesian inference. An earlier study \cite{Li2009Ranking} developed a systematic methodology to assess and rank multiple hazards in a community, and a final example \cite{Cooper2013What} describes the \acrshort{nrc}'s Office of Nuclear Regulatory Research's initial work to support, in part, a multi-hazard Level~2 \acrshort{pra} for \acrshort{lwr}s.

\subsection{Multi-Hazard Model Development}

The multi-hazard PRA model used in this study is based on the EQK-BIN1 configuration, a generic pressurized water reactor (\acrshort{pwr}) model \cite{aras_generic_2024}. The base model represents a seismic event in bin 1, corresponding to a peak ground acceleration (PGA) range of 0.1-0.3\,g, with a representative bin PGA of 0.17\,g. The model includes 610,016 minimal cut sets (\acrshort{mcs}) and, for the base seismic scenario, yields a top event probability of $2.085 \times 10^{-9}$~\cite{batikh_time-dependent_2023}.

To enable multi-hazard analysis, the base EQK-BIN1 model was systematically extended to incorporate hazard interactions, with a particular focus on seismic main shocks and correlated aftershocks. The workflow comprises two sequential stages:

\begin{enumerate}
  \item \textbf{Model construction} with an in-house tool, OpenMHA~\cite{Batikh2024OpenMHA}, resulting in an updated \acrshort{mar-d} file containing combined fault tree/event tree logic and failure data.
  \item \textbf{Model quantification} with SAPHIRE 8, performed either via the \acrshort{gui} or the \acrshort{dll}-based \acrshort{saphsolve} engine.
\end{enumerate}

\paragraph{Model Building with OpenMHA}
OpenMHA is a scenario-based framework designed to facilitate the construction and management of PRA models \cite{batikh_towards_2025_a}. It stores the layout, logic, and \acrshort{ssc} metadata in a MongoDB \cite{MongoDB} instance. The framework includes hazard specific classes, such as \texttt{Fire}, \texttt{Flooding}, \texttt{Seismic}, and \texttt{Tsunami}, which are responsible for assembling the fault-tree logic and interfacing with external physics simulators as needed. After model construction, OpenMHA exports a fully populated MAR-D file that is suitable for use with downstream solvers. In the present study, the focus is on the \texttt{Seismic} class, which is used to explicitly model main shocks and correlated aftershocks. The \acrshort{mar-d} file is imported into SAPHIRE, followed by compilation of a linkage-rule file. Algorithm~\ref{alg:saphire_aftershock_linkage} shows an excerpt that assigns aftershock dependent flags. Section \ref{sec:selected-benchmark-results} summarizes the quantification results obtained from the benchmark study.

\begin{algorithm}[H]
\caption{Excerpt of SAPHIRE linkage rules for the aftershock model}
\label{alg:saphire_aftershock_linkage}
\begin{algorithmic}[1]
\If{SLOCA\_EQ1\_FT \textbf{and} LOOP\_EQ1\_FT}
  \State EVENTREE(EQK\_BIN1) $\gets$ FLAG(ETF\_EQ1\_SL)
\ElsIf{LLOCA\_EQ1\_FT \textbf{and} LOOP\_EQ1\_FT}
  \State EVENTREE(EQK\_BIN1) $\gets$ FLAG(ETF\_EQ1\_LL)
\Else
  \State EVENTREE(EQK\_BIN1) $\gets$ FLAG(ETF\_EQ1)
\EndIf
\If{INIT(IE\_EQ\_BIN1)}
  \State EVENTREE(EQK\_BIN1) $\gets$ ENDSTATE(CD\_EQ)
  \State EVENTREE(EQK\_BIN1) $\gets$ FLAG(T\_1)
\EndIf
\end{algorithmic}
\end{algorithm}