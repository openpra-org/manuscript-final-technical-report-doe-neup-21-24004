\subsection{Probabilistic Risk Assessment Tools and Issues}
\label{sec:pra_tools_history}

In the evolution of \acrshort{pra} during the 1970s, various tools were concurrently developed to facilitate the analysis process. Below is a list of some notable \acrshort{pra} tools:

\begin{itemize}
\item \textbf{PREP and KITT} \cite{18}: Developed as computer code for automatically evaluating a fault tree by Idaho Nuclear Corporation. PREP obtains \acrshort{mcs} of the fault tree using either Monte Carlo or deterministic methods, while KITT obtains numerical probabilities associated with the tree.

\item \textbf{\acrshort{mocus}} \cite{19}: Developed by Aerojet Nuclear Company, a computer program to obtain minimal cut sets from fault trees.

\item \textbf{MODULE} \cite{20}: A computer program capable of handling minimal cut set generation, importance analysis, and uncertainty analysis.

\item \textbf{SIGPI} \cite{21}: A program developed by \acrshort{llnl} for the \acrshort{nrc} that could compute complex systems' probability.

\item \textbf{CAFTA} \cite{22}: Used to develop, maintain, and update single-top failure models using fault tree and event tree modeling techniques. It is currently maintained under the Phoenix Architect Module \cite{23} and is primarily utilized by United States \acrshort{npp} operators.

\item \textbf{SAPHIRE} \cite{24}: A follow-up work of IRRAS \cite{25,26}, developed by the \acrshort{inl} for the \acrshort{nrc}. Released in 1987, it is widely used for performing complete \acrshort{pra} on personal computers and is one of the most used \acrshort{pra} tools in the US.

\item \textbf{RISKMAN} \cite{27}: A personal computer (PC) based, general-purpose, integrated tool for quantitative risk analysis developed by ABS Consulting, Inc \cite{28}.

\item \textbf{KIRAP} \cite{29}: A fault tree construction and analysis code package.

\item \textbf{RiskSpectrum} \cite{30}: An advanced software developed by Relcon Scandpower AB, increasingly used for developing fault trees and event trees to assess system reliability in various parts of \acrshort{npp}.

\item \textbf{RiskA} \cite{31}: An integrated reliability and \acrshort{pra} tool developed by the FDS Team.

\item \textbf{XFTA} \cite{32}: A calculation engine initially developed in 2012 as part of the Open-PSA initiative.

\item \textbf{SCRAM} \cite{33}: An open-source, command-line risk analysis multi-tool currently under enhancement in the PRAG in the Nuclear Engineering Department at \acrshort{ncsu}.

\item \textbf{DeRisk} \cite{34}: A dynamic, uncertain causality graph-embedded risk analysis code package.
\end{itemize}

To summarize, numerous \acrshort{pra} tools have been developed and continuously improved. CAFTA is the preferred choice for most \acrshort{npp} operators in the US. SAPHIRE is the \acrshort{nrc}'s primary software tool for modeling, evaluating, and validating \acrshort{npp} results. Meanwhile, SCRAM is a versatile open-source \acrshort{pra} tool that can perform nearly all \acrshort{pra} calculations.

Although significant improvements have been made, there is still ample opportunity for further enhancements in \acrshort{pra} tools to utilize resources efficiently and obtain accurate results more quickly. Below are some of the current areas requiring improvement~\cite{35}:

\begin{itemize}
\item \textbf{Quantification time and efficiency}
\begin{itemize}
\item Extremely long quantification times may tax the quantification engine, leading to extended waiting periods and potential failures due to memory limitations.
\item Fire \acrshort{pra} models are significantly larger than typical internal-events models, presenting additional quantification challenges.
\end{itemize}

\item \textbf{Dependency analysis for human reliability analysis}

\item \textbf{Model development, maintenance, and updates}
\begin{itemize}
    \item Manual, labor-intensive processes are error-prone and should be automated.
    \item Model manipulation (often done through flag settings in text files) is time-consuming, inefficient, and prone to error.
\end{itemize}

\item \textbf{Risk aggregation}
\begin{itemize}
    \item \textit{Multi-hazard models:}
    \begin{itemize}
        \item Internal-events models often rely on rare-event approximations. These may be inadequate for external events such as earthquakes, high winds, and external flooding, where fragilities of Systems, Structures, and Components are typically higher.
        \item Accurate probability calculations in seismic \acrshort{pra} can quickly exhaust memory if rare-event approximations are not used.
    \end{itemize}
    \item \textit{Multi-unit sites:}
    \begin{itemize}
        \item Many current \acrshort{pra} models assume independence among units at a site, yet in practice, many resources are shared among units.
    \end{itemize}
    \item Analyzing combinations of \acrshort{pra} model elements for criticality.
    \item Uncertainty analysis:
    \begin{itemize}
        \item Most models currently apply only parametric uncertainties, lacking more detailed components.
    \end{itemize}
    \item Communication of risk insights:
    \begin{itemize}
        \item Presenting \acrshort{pra} information to non-\acrshort{pra} practitioners has been a longstanding challenge.
    \end{itemize}
    \item Incorporating new \acrshort{pra} technologies into existing models.
\end{itemize}
\end{itemize}

This project specifically addresses quantification speed, efficiency, model development, maintenance, and updates. Additionally, we discuss risk aggregation, focusing on multi-hazard \acrshort{pra} models.