
\subsection{Automated Synthetic Model Generation}
\begin{longtable}{@{}lllll@{}}
\caption{Configuration options for automatically generating synthetic event \& fault trees.}
\label{tab:autogen_config_et_ft}\\
\toprule
\textbf{Argument} &
  \textbf{Option} &
  \textbf{\begin{tabular}[c]{@{}l@{}}Type\end{tabular}} &
  \textbf{Default} &
  \textbf{\begin{tabular}[c]{@{}l@{}}Applied\\ Option\end{tabular}} \\* \midrule
\endfirsthead
\multicolumn{5}{c}{\textit{Continued: Configuration options for automatically generating synthetic event \& fault trees.}}\\
\toprule
\textbf{Argument} &
  \textbf{Option} &
  \textbf{\begin{tabular}[c]{@{}l@{}}Option\\ Type\end{tabular}} &
  \textbf{Default} &
  \textbf{\begin{tabular}[c]{@{}l@{}}Applied\\ Option\end{tabular}} \\* \midrule
\endhead
%
\bottomrule
\endfoot
%
\endlastfoot
Name for event tree                         & --et-name   & \acrshort{regexp}/string   & autogen & n\_min\_max \\
Number of \acrfull{fe}s                      & --num-func & range, int      & 1     & 1:10:100 \\
Name for fault tree                         & --ft-name   & \acrshort{regexp}/string   & autogen & n\_min\_max \\
Name for the top gate                       & --root      & string          & root    &             \\
Seed for \acrshort{prng}                               & --seed      & int             & 123     & 372         \\
Number of \acrshort{be}s                      & --num-basic & range, int      & 100     & 100:50:5000 \\
Avg. number of gate arguments            & --num-args  & +ve float       & 3.0     &             \\
Weights {[}AND, OR, VOT, NOT, \acrshort{xor}{]} & --weights-g     & +ve float{[}{]}      & {[}1,1,1,1,1{]}  & {[}1,1,1,0,0{]}         \\
Avg. \% of common \acrshort{be}s per gate     & --common-b  & +ve float       & 0.3     &             \\
Avg. \% of common gates per gate            & --common-g  & +ve float       & 0.1     &             \\
Avg. number parents for common \acrshort{be}s  & --parents-b & +ve float       & 2       &             \\
Avg. number of parents for common gates  & --parents-g & +ve float           & 2       &             \\
Number of gates (\textless{}= 0 means auto) & --num-gate  & int             & 0       &             \\
Maximum probability of \acrshort{be}s         & --max-prob & float {[}0,1{]} & 0.5     & 0.05        \\
Minimum probability of \acrshort{be}s         & --min-prob  & float {[}0,1{]} & 0.5     & 0.01        \\
Number of \acrfull{he}s                     & --num-house & int             & 0       &             \\
Number of \acrshort{ccf} groups                        & --num-ccf   & int             & 0       &             \\
Output format {[}\acrshort{xml}, \acrshort{json}, \acrshort{jsinp}{]}               & --out       & \{xml, jsinp, json\}   & xml     & xml, jsinp   \\* \bottomrule
\end{longtable}

\cite{Farag2023Benchmarking}
% The key to generating models for quantification engines lies in creating identical models for accurate comparisons. However, when applying the provided methodology, users may encounter unfamiliarity with the specific PRA tool. Synthetic models are utilized for the case study to address this challenge and gain insights into complex sources. Synthetic models offer flexibility in identifying complexity origins and assessing the impact on resource consumption during quantification.
% As shown in Figure 12, model generation involves combining model logic with failure data for basic events. Each component of the model generation process is essential for enhancing PRA tools. Model logic is critical because the quantification engine derives cut sets from it. At the same time, failure data is fundamental for calculating key risk metrics, such as top event probability, importance measures, and uncertainty estimates.
 
% Figure 12: Model generator framework.
% The main components of the model generation utility are illustrated in Figure 13. Users define model parameters through a configuration file, the input for generating synthetic models. If a schema is available for the selected model format, the generated models are validated against it. Finally, the verified models are solved using an appropriate quantification engine.
 
% Figure 13: Model generator utility.
% Table 2. PRA model generator arguments with case study configurations.
% #	Argument	Options	Configurations
% 1	Name for fault tree	--ft-name	Autogenerated
% 2	Name for the root gate	--root	Root
% 3	Seed for PRNG	--seed	123
% 4	Number of basic events	--num-basic	100:50:5000
% 5	Average number of gate arguments	-num-args	3.0
% 6	Weights for gates [AND, OR, K/N, NOT, XOR]	--weights-g	[1,1,1,0,0]
% 7	Average percentage of common basic events per gate	--common-b	0.3
% 8	Average percentage of common gates per gate	--common-g	0.1
% 9	Average number of parents for common basic events	--parents-b	2
% 10	Average number of parents for common basic events	--parents-g	2
% 11	Number of gates	-g or -num-gate	0
% 12	Maximum probability of basic events	---max-prob	0.05
% 13	Minimum probability of basic events	--min-prob	0.01
% 14	Number of house events	--num-house	0
% 15	Number of common-cause-failure groups	--num-ccf	0
% 16	A file to write the fault tree	-o or -out	XML or JSInp
% 17	Apply the other format to the output	--other	--saphsolve

% The synthetic models for this section have been generated using a fault tree generator, relying on seventeen essential parameters outlined in Table 2. Careful configuration of fault trees, as specified in the fourth column, ensures diverse models are constructed. The number of basic events is systematically adjusted, ranging from 100 to 5,000 with 50-event increments, resulting in 99 models for each engine. Each gate within the fault trees—whether an AND, OR, or K/N gate—is randomly selected, with each gate type's average occurrence expected to match one another closely.
% The configuration also considers common basic events, with probabilities chosen judiciously to strike a balance between avoiding values that trigger premature truncation and steering apparent of overly large values that contravene the rare-event approximation.
% It is important to note a significant modeling difference between SCRAM and SAPHSOLVE: SCRAM accepts truncation parameters or values via the command line, whereas SAPHSOLVE accepts them inside the input file. A probability truncation value of 10-20 is used, while no size truncation is applied for either tool.
% Additionally, common cause failures, uncertainty analysis, and importance measurement are intentionally excluded from this study and reserved for future work. The SAPHSOLVE [38] and SCRAM [67] models utilized for this work are accessible through the repositories.
% Alongside the fault trees, synthetic event trees can also be generated using the event tree generator. To generate synthetic event trees, the user must provide a comma-separated values (CSV) formatted model configuration file as the input. Users can specify up to 22 arguments including naming convention, fault tree logic, and model data, as shown in Table 3. The table also provides default values for each argument, ensuring alignment with the expected data types. The first two arguments define the event tree structure, while the remaining arguments—except for the output file path—pertain to fault tree logic and data. All implementations related to the event tree generator have been written in Python.
% Each run generates a single event tree initiated by a single initiating event and linked to multiple fault trees. The model generation process begins by determining the number of functional events. The model generator utility creates a separate fault tree logic and dataset for each functional event within the same input file structure. 
% By default, the first fault tree consists of 100 basic events, with each subsequent fault tree increasing by 50 additional basic events. This approach ensures that sequences share common basic events. 
% The event tree includes both a success and a failure branch for each functional event, each connected to the corresponding fault tree. Users configure the fault tree logic and data through the CSV file.  The model generation process concludes by exporting the final model as an Extensible Markup Language (XML) file. 
% Open-PSA has a well-defined schema based on RELAX NG [68], which enables the verification of generated models against a standardized format. The verification process is implemented separately, as it is not limited to validating only generated models but can be applied to any model in the Open-PSA format. 
% The schema file is parsed and loaded using our methodology for verification purposes. The process generates a log file containing three columns: XML File, Status, and Errors. The Status column indicates whether the model is Valid or Invalid. If a model is deemed Invalid, the corresponding error messages are recorded in the log file, which is stored in CSV format. If errors are encountered, the model generator utility is reviewed and updated to resolve the issues.
% Table 3: Model configuration arguments.
% # 	Argument 	Expected Type 	Default 
% 1 	Event Tree Name 	String 	Generated-Event-Tree 
% 2 	Number of Functional Events 	Integer 	1 
% 3 	Seed for Fault Tree Generation 	Integer 	123 
% 4 	Number of Basic Events for First Fault Tree 	Integer 	100 
% 5 	Average Number of Gate Arguments 	Float 	3.0 
% 6 	Weights for AND Gate 	Integer 	1 
% 7 	Weights for OR Gate 	Integer 	1 
% 8 	Weights for K out of N Gate 	Integer 	1 
% 9 	Weights for NOT Gate 	Integer 	0 
% 10 	Weights for XOR Gate 	Integer 	0 
% 11 	Average Percentage of Common Basic Events per Gate 	Float 	0.1 
% 12 	Average Percentage of Common Gates per Gate 	Float 	0.1 
% 13 	Average Number of Parents for Common Basic Events 	Float 	2.0 
% 14 	Average Number of Parents for Common Basic Gates 	Float 	2.0 
% 15 	Number of Gates (discards 13 and 14 if not zero) 	Integer 	0 
% 16 	Maximum Probability for Basic Events 	Float 	0.1 
% 17 	Minimum Probability for Basic Events 	Float 	0.01 
% 18 	Number of House Events 	Integer 	0 
% 19 	Number of Common Cause Failure Groups 	Integer 	0 
% 20 	Size of Common Cause Failure Group 	Integer 	0 
% 21 	Model for Common Cause Failure 	String 	alpha-factor 
% 22 	Output File Path 	String 	./../models/openpsa 

