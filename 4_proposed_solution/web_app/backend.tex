\subsection{Backend \acrshort{rest} \acrshort{api}}

The OpenPRA backend is a Node.js based \acrshort{rest} \acrshort{api} that coordinates all qualitative \acrshort{pra} workflows, exposing \acrfull{crud} endpoints for every element defined by the OpenPRA \acrfull{mef} schema. Contrary to the distributed job system's \acrshort{rest} \acrshort{api}, which is an independent microservice dedicated to quantification operations only, the backend \acrshort{rest} \acrshort{api} is tightly integrated with the web front‑end and the distributed databases. Thus, computational requests are routed to the job broker’s \acrshort{api}, while model editing, security, and user management operations are handled by the backend \acrshort{api}. All client requests from the front-end go through this \acrshort{api}, which handles user authentication, authorization, and persists model data in the distributed databases. In practice, each \acrshort{mef} namespace has a corresponding set of \acrshort{api} routes. For example, endpoints such as \texttt{`/api/fault-trees`} or \texttt{`/api/event-trees`} allow clients to create, read, update, or delete the graph-structured \acrshort{pra} models.  The front‑end data is first validated by the Nestia runtime library against the OpenPRA \acrshort{mef} \acrshort{json} schema and then they are saved as \acrfull{bson} documents inside the database.

The backend implements full user and collaboration management.  It provides endpoints for user registration, login (issuing JWT tokens), password verification, and user profile or solver preferences.  It secures these endpoints using JWT/OAuth2 guards and role-based access control. Role and invitation endpoints allow administrators to create new roles, assign permissions, and manage team collaboration (including sending or verifying invitation tokens for new users).  

Table \ref{tab:openpra-backend-endpoints} lists the main \acrshort{api} entry points (grouped by function), with the \acrshort{http} request methods they support and a brief description of each endpoint’s purpose:

\begin{landscape}
\begin{longtable}{@{}lll@{}}
\caption{Main REST API entrypoints exposed by the OpenPRA backend.}
\label{tab:openpra-backend-endpoints}\\
\toprule
\textbf{Endpoint} & \textbf{Methods} & \textbf{Description} \\
\midrule
\endfirsthead
\multicolumn{3}{c}{\textit{Continued: Main REST API entrypoints exposed by the OpenPRA backend.}}\\
\toprule
\textbf{Endpoint} & \textbf{Methods} & \textbf{Description} \\
\midrule
\endhead
\bottomrule
\endfoot
%
\multicolumn{3}{@{}l}{\textbf{Authentication}}\\
\midrule
\texttt{/api/auth/token-obtain} & POST & Issue JWT token (user login) \\
\texttt{/api/auth/verify-password} & POST & Verify user credentials/password \\
\addlinespace
\multicolumn{3}{@{}l}{\textbf{Collaboration (Users)}}\\
\midrule
\texttt{/api/collab/user} & GET, POST & List users; create new user \\
\texttt{/api/collab/user/\{user\_id\}} & GET, PUT & Get or update a user by ID \\
\texttt{/api/collab/user/\{user\_id\}/preferences} & GET, PUT & Get or update user preferences \\
\texttt{/api/collab/validateEmail} & POST & Check email availability/validity \\
\texttt{/api/collab/validateUsername} & POST & Check username uniqueness \\
\addlinespace
\multicolumn{3}{@{}l}{\textbf{Collaboration (Invites \& Roles)}}\\
\midrule
\texttt{/api/collab/invite} & POST, PUT & Create or update an invitation \\
\texttt{/api/collab/invites} & GET & List all invitations \\
\texttt{/api/collab/invite/\{id\}} & GET, DELETE & Get or cancel invite by ID \\
\texttt{/api/collab/verify-invite} & POST & Verify/redeem an invite token \\
\texttt{/api/collab/roles} & GET, POST, PUT & List, create, or update access roles \\
\texttt{/api/collab/roles/\{roleId\}} & DELETE & Delete a role by ID \\
\addlinespace
\multicolumn{3}{@{}l}{\textbf{Model Graphs}}\\
\midrule
\texttt{/api/model/fault-tree-graph} & GET, POST & Create or retrieve fault-tree graph structure \\
\texttt{/api/model/event-tree-graph} & GET, POST & Create or retrieve event-tree graph structure \\
\texttt{/api/model/event-sequence-diagram-graph} & GET, PATCH & Get or update event-sequence diagram graph \\
\texttt{/api/model/event-sequence-diagram-graph/update-label} & PATCH & Update labels on the event-sequence graph \\
\addlinespace
\multicolumn{3}{@{}l}{\textbf{\acrfull{fmea}}}\\
\midrule
\texttt{/api/model/fmea/\{id\}} & GET & Retrieve FMEA table by ID \\
\texttt{/api/model/fmea/\{id\}/row} & PUT & Add or update a row in the FMEA \\
\texttt{/api/model/fmea/\{id\}/column} & PUT & Add or update a column in the FMEA \\
\texttt{/api/model/fmea/\{id\}/cell} & PUT & Update a cell value in the FMEA \\
\texttt{/api/model/fmea/\{id\}/dropdown} & PUT & Update dropdown options in the FMEA \\
\texttt{/api/model/fmea/\{id\}/delete} & PUT & Delete the entire FMEA table \\
\texttt{/api/model/fmea/\{id\}/column/updateName} & PUT & Rename a FMEA column \\
\texttt{/api/model/fmea/\{id\}/column/updateType} & PUT & Change a column’s type \\
\texttt{/api/model/fmea/\{fmeaid\}/\{rowid\}/delete} & DELETE & Delete a specific FMEA row \\
\addlinespace
\multicolumn{3}{@{}l}{\textbf{Typed Models (Hazard Tables)}}\\
\midrule
\texttt{/api/model/internal-events} & GET, POST, DELETE & Manage internal-event entries \\
\texttt{/api/model/internal-events/\{id\}} & GET, PATCH & Get or update an internal-event by ID \\
\texttt{/api/model/internal-hazards} & GET, POST, DELETE & Manage internal-hazard entries \\
\texttt{/api/model/internal-hazards/\{id\}} & GET, PATCH & Get or update an internal-hazard by ID \\
\texttt{/api/model/external-hazards} & GET, POST, DELETE & Manage external-hazard entries \\
\texttt{/api/model/external-hazards/\{id\}} & GET, PATCH & Get or update an external-hazard by ID \\
\texttt{/api/model/full-scope} & GET, POST, DELETE & Manage full-scope analysis entries \\
\texttt{/api/model/full-scope/\{id\}} & GET, PATCH & Get or update a full-scope entry by ID \\
\addlinespace
\multicolumn{3}{@{}l}{\textbf{Other PRA Components}}\\
\midrule
\texttt{/api/model/initiating-events} & POST & Add initiating events to a model \\
\texttt{/api/model/markov-chains} & POST & Add Markov chain models \\
\texttt{/api/model/weibull-analysis} & POST & Add Weibull failure-rate data \\
\texttt{/api/model/risk-integration} & POST & Add risk integration parameters \\
\texttt{/api/model/radiological-consequence-analysis} & POST & Add consequence-model data \\
\texttt{/api/model/mechanistic-source-term} & POST & Add source-term model data \\
\texttt{/api/model/event-sequence-quantification-diagram} & POST & Add event-sequence quantification diagrams \\
\texttt{/api/model/data-analysis} & POST & Add data-analysis entries \\
\texttt{/api/model/human-reliability-analysis} & POST & Add HRA entries \\
\texttt{/api/model/systems-analysis} & POST & Add systems analysis entries \\
\texttt{/api/model/success-criteria} & POST & Add success-criteria entries \\
\texttt{/api/model/event-sequence-analysis} & POST & Add event-sequence analysis entries \\
\texttt{/api/model/operating-state-analysis} & POST & Add operating-state analysis entries \\
\addlinespace
\multicolumn{3}{@{}l}{\textbf{Quantification}}\\
\midrule
\texttt{/api/quantify/with-scram-binary} & POST & Submit a model to the SCRAM engine via job queue \\
\end{longtable}
\end{landscape}

