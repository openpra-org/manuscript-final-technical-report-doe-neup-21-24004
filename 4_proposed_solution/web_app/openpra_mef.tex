\subsection{OpenPRA Model Exchange Format}

A standardized \acrfull{mef} is critical for collaborative development, model reuse, and interoperability among \acrshort{pra} tools.  By defining a common \acrshort{json} schema, users and different tools can exchange \acrshort{pra} models without ambiguity.  \acrshort{json} was chosen over \acrshort{xml} to align with modern web technologies and simplify data handling in JavaScript/TypeScript based applications. \acrshort{json} is more compact, easier to validate, and integrates directly with the front-end and backend code.  To preserve continuity with existing \acrshort{pra} models, the format remains compatible with the legacy OpenPSA \acrshort{mef} \acrshort{xml} format via automated conversion.

The \acrshort{mef} schema provides a single, consistent definition of \acrshort{pra} models for both the OpenPRA front-end and backend. All user input in the web editor must conform to this schema, and the backend validates the submitted \acrshort{json} against it before processing ensuring alignment with the Non-\acrshort{lwr} \acrshort{pra} technical elements specified in \acrshort{nrc} Reg Guide 1.247.

The \acrshort{mef} schema is organized into namespaces corresponding to each of the Reg Guide 1.247 technical elements.  Each namespace defines the specific entities and attributes for that element, along with references linking it to other elements. The major namespaces include:

\begin{itemize}
  \item \textbf{Plant operating states analysis}: It defines each operating state with attributes such as power level, reactor coolant system configuration, equipment status, barrier status etc.  This namespace lists each state as a structured record.  It supports traceability by linking each state to applicable initiating events and to success criteria via ID references, ensuring that every scenario is evaluated under the correct operating conditions.
  \item \textbf{Initiating event analysis}: It defines initiating event families and individual events (internal or external hazards), along with their estimated frequencies.  Entities include grouped event definitions with triggers and fault modes.  This namespace enforces consistency by linking events to the relevant operating states (each event is associated with a POS) and by requiring presence of all mandated event categories, which ensures a complete and consistent event definition.
  \item \textbf{Event sequence analysis}: It defines event sequences (analogous to event trees) that capture the progression of system successes and failures.  Entities include sequence definitions with branching logic and outcome states.  The schema supports traceability by connecting each sequence to its initiating event, its associated success criteria, and its end states, enabling end-to-end linkage of the sequence logic across \acrshort{pra} elements.
  \item \textbf{Success criteria development}: It defines the success criteria for safety functions and systems given specific initiating events.  Entities include criteria records that list the required systems, components, and actions needed to meet each safety function.  The schema ties each success criterion to corresponding system or component IDs and to event sequences, ensuring that every criterion is linked to the relevant events and system configurations.
  \item \textbf{Systems analysis}: It defines system models including components, failure modes, subsystem dependencies, and common-cause failures.  Entities include system topology and fault tree structures for system unavailability.  It supports quantification by explicitly declaring component IDs and failure relationships, enabling automated checking of dependency effects (e.g.\ CCFs) and completeness of the system model.
  \item \textbf{Human reliability analysis}: Implements the Human Reliability Analysis element.  It defines \acrfull{hfe}s, task descriptions, and recovery actions.  Entities include \acrshort{hfe} definitions linked to plant states, system conditions, and operator actions.  It provides traceability by associating each \acrshort{hfe} with the specific plant state and system context in which it can occur, ensuring that all human errors are connected to the relevant event sequences.
  \item \textbf{Mechanistic source term analysis}: It defines release categories and source-term parameters for radioactive material releases.  Entities include source-term models (e.g.\ fractions of radionuclides released) tied to specific sequences and physical barriers.  The schema supports quantification by linking each release category back to its triggering event sequence and to the success criteria, enabling consistent calculation of source terms.
  \item \textbf{Radiological consequence analysis}: It defines consequence calculation scenarios, dose endpoints, and transport pathways.  Entities include consequence cases that map release quantities to dose or risk metrics.  It captures dependencies by connecting each consequence case to the corresponding release category from the source-term analysis and to plant conditions, so that material releases are systematically propagated to dose outcomes.
  \item \textbf{Risk integration}: It defines the aggregated \acrshort{pra} model, including top events, cut sets, and risk metrics.  Entities include combined risk data structures and summary outputs.  It provides end-to-end traceability by linking each calculated frequency or risk value back to the contributing initiating events, sequences, systems, and success criteria, ensuring that the entire \acrshort{pra} chain is connected.
\end{itemize}
