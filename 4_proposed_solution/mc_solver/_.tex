\chapter{Building a Data-Parallel Monte-Carlo Probability Estimator}

To handle massively parallel Monte Carlo evaluations of large-scale Boolean functions, we have developed a preliminary layered architecture that organizes computation in a topological graph. At the lowest level, each Boolean variable/basic event (e.g., a component failure) is associated with a random number generator to sample its truth assignment. We bit-pack these outcomes—storing multiple Monte Carlo samples in each machine word—to maximize computational throughput and reduce memory footprint. Subsequent layers consist of logically higher gates or composite structures that receive the bit-packed results from previous layers and combine them in parallel using coalesced kernels. By traversing the computation graph topologically, dependencies between gates and events are naturally enforced, so kernels for each layer can run concurrently once all prerequisite layers finish, resulting in high kernel occupancy and predictable throughput.
In practice, each layer is dispatched to an accelerator node using a data-parallel model implement using \acrshort{sycl}. The random number generation pipelines are counter-based, ensuring reproducibility and thread-safety even across millions or billions of samples. Gates that go beyond simple AND/OR logic—such as K-of-N operators—are handled by specialized routines that can exploit native popcount instructions for efficient threshold evaluations. As we progress upwards through the layered topology, each gate or sub-function writes out its bit-packed output, effectively acting as an input stream to the next layer.
Throughout the simulation, online tallying kernels aggregate how often each node or gate evaluates to True. These tallies can then be turned into estimates of probabilities and sensitivity metrics on the fly. This approach also makes adaptive sampling feasible: if specific gates appear to dominate variance or are tied to particularly rare events, additional sampling can be allocated to their layer to refine estimates.

\section{PRA Models as Probabilistic Directed Acyclic Graphs (PDAGs)}
\label{sec:unified_pra_dag}

Up to this point, we have introduced \acrshort{et}s to capture the forward evolution of scenarios and \acrshort{ft}s to capture the top-down decomposition of system failures. In a full-scale \acrshort{pra}, many \acrshort{et}s and \acrshort{ft}s are linked to form a single overarching model. The goal is to represent:
\begin{enumerate}
\item The branching structure of multiple event trees, which may feed into one another (\acrshort{et}\(\to\)\acrshort{et}),  
\item Multiple fault trees that themselves can reference or be referenced by other \acrshort{ft}s (\acrshort{ft}\(\to\)\acrshort{ft}),  
\item Event trees that invoke fault trees to quantify key failure probabilities (\acrshort{et}\(\to\)\acrshort{ft}), and similarly fault trees whose outcome may direct the next branch or state in an event tree.  
\end{enumerate}
All of these interconnections can be consolidated into a single \acrshort{pdag}.  In broad terms, its nodes stand for either (i)~\acrshort{et} nodes (initiating or functional events), (ii)~\acrshort{ft} gates or intermediate events, or (iii)~basic events.

\subsection{Basic Structure and Notation}
Let us denote:
\begin{itemize}
\item \(\{\Gamma_1, \Gamma_2, \dots, \Gamma_{M}\}\) as the \emph{collection of event trees}, where each \(\Gamma_{i}\) may represent a different initiating event or system phase.  Every \(\Gamma_i\) is itself structured as in Section~\ref{sec:event_tree_definition}, with a set of node events and directed edges (success/failure branches).  
\item \(\{\Phi_{1}, \Phi_{2}, \dots, \Phi_{N}\}\) as the \emph{collection of fault trees}, each built according to Section~\ref{sec:fault_tree_definition}.  Every \(\Phi_j\) has a unique top event, an acyclic arrangement of gates (AND, OR, voting, etc.), and a set of basic events.  
\end{itemize}

In a large \acrshort{pra}, any \acrshort{et} \(\Gamma_i\) may:
\begin{itemize}
\item Lead to another \acrshort{et} \(\Gamma_{k}\) under certain branch outcomes (\acrshort{et}\(\to\)\acrshort{et}).  
\item Include a branch that requires computing ``System \(X\) fails'' via a fault tree \(\Phi_j\) (\acrshort{et}\(\to\)\acrshort{ft}).  
\end{itemize}
Similarly, a fault tree \(\Phi_{j}\) may:
\begin{itemize}
\item Contain the top event of another fault tree \(\Phi_{k}\) as one of its inputs (\acrshort{ft}\(\to\)\acrshort{ft}).  
\item Generate an outcome (e.g.\ subsystem fails) that triggers a branch in some event tree \(\Gamma_i\).  
\end{itemize}
These inter-dependencies can be organized into a single \acrshort{pdag}, denoted
\[
\mathcal{M} \;=\; \bigl(\mathcal{V},\,\mathcal{A}\bigr),
\]
where:
\begin{itemize}
\item \(\mathcal{V}\) is the set of \emph{all} nodes in the unified model.  Each node \(v\in \mathcal{V}\) has a type indicating whether it belongs to an event tree (\acrshort{et}-node), a fault tree (\acrshort{ft}-gate), or is a \acrfull{be}.  
\item \(\mathcal{A}\subseteq \mathcal{V}\times \mathcal{V}\) is the set of directed edges.  Each \((u,v)\in \mathcal{A}\) signifies a logical or probabilistic dependency from node \(u\) to node \(v\).  
\end{itemize}
By design, \(\mathcal{M}\) is acyclic: no path loops back through the same node.  This condition prevents paradoxical definitions of probabilities or statements (e.g., an event that depends on itself).

\subsection{Nodes and Their Inputs}

We partition \(\mathcal{V}\) into three principal categories:
\begin{enumerate}
\item \textbf{Basic Events (BEs).}  Let \(\mathcal{B}\subset \mathcal{V}\) be the set of all basic events across all \acrshort{ft}s.  Each \(b\in\mathcal{B}\) is associated with a failure probability \(p(b)\in [0,1]\).  A node \(b\) in the \acrshort{pdag} has no incoming edges (i.e., it is a leaf source for the rest of the logic).  

\item \textbf{Fault Tree (\acrshort{ft}) Nodes.}  Let \(\mathcal{G}\subset \mathcal{V}\) be the set of \emph{internal \acrshort{ft}-gates or intermediate \acrshort{ft}-events}\footnote{In some treatments, all \acrshort{ft} internal nodes are called gates, even if no actual gate symbol is used.  We adopt the broader term \emph{\acrshort{ft}~nodes} for uniformity.}, unified across \(\{\Phi_1,\dots,\Phi_N\}\).  
    \begin{itemize}
    \item Each \acrshort{ft} node \(g\in \mathcal{G}\) has one output event \(g\) (the node itself in the \acrshort{pdag}).  
    \item The set of inputs to \(g\) may include basic events \(\mathcal{B}\) (e.g., component failures) and/or other \acrshort{ft} nodes \(\mathcal{G}\) (subsystem-level events). If \(g\) is the top event of \(\Phi_j\), then it may appear as an input into a \emph{different} \acrshort{ft}’s gate or an \acrshort{et} node.  
    \item As with standard \acrshort{ft} logic, each gate has a type in \(\{\mathrm{AND},\,\mathrm{OR},\,\mathrm{VOT}(k/n),\,\ldots\}\).  Its output (failure state) is a Boolean function of its inputs’ failure states.
    \end{itemize}

\item \textbf{Event Tree (\acrshort{et}) Nodes.}  Let \(\mathcal{E}\subset \mathcal{V}\) be the set of \emph{\acrshort{et}-nodes}, each referring either to an initiating event (IE) or to a functional event within some event tree \(\Gamma_i\).  
    \begin{itemize}
    \item An \acrshort{et} node \(e\) in \(\Gamma_i\) can have multiple outgoing edges, each labeled by a particular outcome (e.g., success/failure).  These edges may lead to another \acrshort{et} node (continuing the same event tree), to the root/top event of a fault tree (to evaluate subsystem reliability), or even to the root of a \emph{different} event tree (\acrshort{et}\(\to\)\acrshort{et}).
    \item As in Section~\ref{sec:event_tree_definition}, each outgoing branch from \(e\) has an associated conditional probability, conditioned on the event \(e\) itself having occurred.  
    \end{itemize}
\end{enumerate}
Hence, 
\[
\mathcal{V} \;=\;
\underbrace{\mathcal{B}}_{\text{basic events}}
\;\;\cup\;\;
\underbrace{\mathcal{G}}_{\text{\acrshort{ft} nodes}}
\;\;\cup\;\;
\underbrace{\mathcal{E}}_{\text{\acrshort{et} nodes}},
\]
and every node \(v\) in \(\mathcal{V}\) has an \emph{input set}
\[
I(v)
\;\;\subseteq\;\;
\mathcal{B}\,\cup\,\mathcal{G}\,\cup\,\mathcal{E}
\;\;=\;\;\mathcal{V}\setminus\{v\},
\]
indicating which nodes feed into \(v\).  By the \acrshort{pdag} property, \(v\) cannot be an ancestor of itself.

\subsection{Edge Types and Probability Assignments}

Each directed edge \((u,v)\in \mathcal{A}\) belongs to one of several categories:

\begin{itemize}
\item \textbf{\acrshort{et} \(\to\) \acrshort{et} edges. [Transfers]}  These edges connect an \acrshort{et} node \(u\in\mathcal{E}\) to another \acrshort{et} node \(v\in\mathcal{E}\) within the same tree \(\Gamma_i\) or leading to a subsequent tree \(\Gamma_{k}\).  In typical diagrams, these edges denote if \(u\) occurs, then with probability \(\theta_{u\to v}\) we transition to \(v\).  Probabilities on all child edges of \(u\) sum to 1, reflecting the partition of possible outcomes.

\item \textbf{\acrshort{et} \(\to\) \acrshort{ft} edges. [Functional Events]}  These edges represent the case where an \acrshort{et} node \(u\in\mathcal{E}\) designates Check if subsystem \(\Phi_j\) has failed.  Formally, the next node \(v\in\mathcal{G}\) is the top event (or relevant subsystem event) in fault tree \(\Phi_j\).  The probability of \(v\) failing is not assigned directly on the edge but is computed via the logical structure of \(\Phi_j\).  

\item \textbf{\acrshort{ft} \(\to\) \acrshort{ft} edges. [Transfer Gates]}  Such edges arise when the top event (or an intermediate gate) \(u\in\mathcal{G}\) of one fault tree is input to a gate \(v\in\mathcal{G}\) in another fault tree.  For instance, if \(\Phi_{1}\) captures the failure mode of a pump and \(\Phi_{2}\) captures the failure of a coolant subsystem that includes that same pump’s top event.  

\item \textbf{\acrshort{ft} \(\to\) \acrshort{et} edges. [Initiating Events]}  Less common but still possible are edges that carry an outcome of a fault tree node \(u\in\mathcal{G}\) to an \acrshort{et} node \(v\in\mathcal{E}\).  For instance, an initiating event might depend on whether a certain subsystem fails, as computed by a separate \acrshort{ft} \(\Phi_j\). Support system \acrshort{ft}s are one such example.

\item \textbf{BEs as sources (no incoming edges).}  Each basic event \(b\in\mathcal{B}\) has probability \(p(b)\) of failing, so \(\Pr[X_b=1]\!=\!p(b)\).  These do not have incoming edges because they represent fundamental failure modes, not dependent on other events within the model.  
\end{itemize}
Denote by \(\theta_{u\to v}\) the \emph{conditional probability} weighting an \acrshort{et}-type edge \((u\!\to\!v)\).  If node \(u\) splits into children \(v_1,\dots,v_k\), then
\[
\sum_{i=1}^k \theta_{u \to v_i} \;=\; 1,
\quad
\theta_{u \to v_i}\;\; \ge\;0.
\]
By contrast, \acrshort{ft}-type edges do not carry numerical probabilities directly.  Instead, a gate node \(v\in \mathcal{G}\) aggregates its inputs’ \emph{fail/success states} via Boolean logic (AND, OR, \(\mathrm{VOT}(k/n)\), etc.) to yield \(\pi_{\mathcal{M}}(S,v)\in\{0,1\}\), the node \(v\)’s failure state under a set \(S\) of basic-event failures.

\subsection{Semantics of the Unified Model}

A full \emph{scenario} in \(\mathcal{M}\) extends from a designated \textit{initial node} (often an initiating event \(I\in \mathcal{E}\)) forward through whichever \acrshort{et} or \acrshort{ft} edges are triggered.  Because no cycles exist, every path eventually terminates in either (a)~an \acrshort{et} leaf (end-state), (b)~a top event that is not expanded further, or (c)~a final subsystem outcome deemed not to propagate further risk.

\subsubsection{Failure States in the Fault Trees.}
For any subset \(S\subseteq \mathcal{B}\) of basic events that fail:
\begin{enumerate}
\item Each \(b\in \mathcal{B}\) fails iff \(b\in S\).  
\item Each \acrshort{ft} node \(g\in \mathcal{G}\) has failure state \(\pi_{F}(S,g)\) determined by the usual fault tree semantics (Section~\ref{sec:fault_tree_definition}).  
\end{enumerate}
That is, a node \(g\) in a fault tree \(\Phi_j\) is in failure mode when its logical gate type indicates failure is activated by the failures of its inputs (which might be other gates or basic events).

\subsubsection{Branching in the Event Trees.}
Whenever an \acrshort{et} node \(e\in \mathcal{E}\) is reached, outgoing edges \(\{(e \to e_1), (e\to e_2),\dots\}\) partially partition the scenario space.  The choice of which child \(e_i\) is realized is probabilistic, with probabilities \(\theta_{e\to e_i}.\)

\subsubsection{Event-Tree to Fault-Tree Links.}
If an edge \((e\to g)\) connects an \acrshort{et} node \(e\) to a \emph{fault tree top event} \(g\in \mathcal{G}\), the scenario path triggers the question Does \(g\) fail?  The probability that \(g\) is in failure, conditional on having arrived at node \(e\), is determined by the set \(S\subseteq \mathcal{B}\) of basic events that happen to fail in that scenario plus the gate logic of \(\Phi_j\).

Altogether, scenario outcomes in \(\mathcal{M}\) thus combine:
\begin{itemize}
\item \(\mathbf{X}_{\mathcal{B}} = \{\,X_b : b\in\mathcal{B}\}\), where \(X_b\in\{0,1\}\) indicates whether basic event \(b\) fails or not, and  
\item A chain of \acrshort{et} decisions or fault-tree outcomes, traveling through the \acrshort{pdag} until reaching a terminal node.  
\end{itemize}
If all basic events are assumed independent with probabilities \(\{p(b)\}\), then the \emph{global} likelihood of a specific path \(\omega\) from an initiating event \(I\) to a final outcome (and with a particular pattern of success/failure across \(\mathcal{B}\)) factors into products of:
\begin{enumerate}
\item The product of \(\prod_{b\in S}p(b)\,\times\,\prod_{b\notin S}[\,1-p(b)\,]\) for the relevant \(S\subseteq\mathcal{B}\).  
\item The product of all \acrshort{et}-edge probabilities \(\theta_{u\to v}\) encountered along the path (\(u\in\mathcal{E}\)).  
\item The logical constraints from each visited fault tree node \(g\in \mathcal{G}\), which impose \(\pi_{F}(S,g)\in\{0,1\}\) in or out of failure.  
\end{enumerate}
Summing over all valid paths (or equivalently over all subsets \(S\subseteq \mathcal{B}\) and the result of each \acrshort{et}/\acrshort{ft} branching) yields the total system risk measure, such as the probability of a severe radiological release, or the probability that a certain undesired top event emerges.

\subsection{Formal Definition of the Unified Model}

Bringing these elements together, we propose the following definition:

\begin{definition}[Unified \acrshort{pra} Model]
\label{def:unified_pra_dag}
A \emph{unified \acrshort{pra} model} is a \acrfull{pdag}
\[
\mathcal{M}
\;=\;
\Bigl\langle\,
  \mathcal{V}\;=\;\mathcal{B}\cup\mathcal{G}\cup\mathcal{E},
  \;\;\mathcal{A},
  \;\;p(\cdot),
  \;\;\pi_{F}
\Bigr\rangle
\]
with the following properties:

\begin{enumerate}
\item \(\mathcal{B}\) is the set of \textbf{basic events}, each \(b\in \mathcal{B}\) failing with probability \(p(b)\).  These nodes have no incoming edges in \(\mathcal{M}\).  
\item \(\mathcal{G}\) is the set of \textbf{\acrshort{ft} nodes}, each representing a gate or intermediate event in a fault tree.  For \(g\in \mathcal{G}\), the function
\[
\pi_{F}(S,g)\;=\;\begin{cases}
1, & \text{if fault-tree logic declares \(g\) fails under \(S\subseteq\mathcal{B}\)},\\
0, & \text{otherwise}.
\end{cases}
\]
\item \(\mathcal{E}\) is the set of \textbf{\acrshort{et} nodes}, each having zero or more outgoing edges.  If node \(e\) has children \(\{v_1,\dots,v_k\}\subset \mathcal{V}\), then the edges \(\{(e\to v_i)\}_{i=1}^k\) carry probabilities \(\theta_{e\to v_i}\ge 0\) summing to 1.  
\item \(\mathcal{A}\subseteq (\mathcal{V}\times \mathcal{V})\) is the set of directed edges.  An edge \((u\to v)\) can be:
  \begin{itemize}
  \item \(\mathrm{\acrshort{et}\to \acrshort{et}}\):  event-tree branching,  
  \item \(\mathrm{\acrshort{et}\to \acrshort{ft}}\):  an event tree referencing a fault tree’s top event,  
  \item \(\mathrm{\acrshort{ft}\to \acrshort{ft}}\):  linking one fault tree node to another’s input,  
  \item \(\mathrm{\acrshort{ft}\to \acrshort{et}}\):  an \acrshort{ft} outcome passed back to an event tree node,  
  \item \(\mathrm{\acrshort{be}}\to\varnothing\):  no edges emanate from a basic event node.  
  \end{itemize}
\item The graph \(\mathcal{M}\) is acyclic: no path in \(\mathcal{A}\) returns to a previously visited node.  
\end{enumerate}
\end{definition}

In practice, we anticipate large nuclear \acrshort{pra}s to contain hundreds of event trees and thousands of fault trees, sharing many basic events or subsystem-level fault trees.  By embedding them in \(\mathcal{M}\), one can systematically compute probabilities for any high-level risk measure (e.g., core damage, large release) by enumerating scenario paths or using specialized algorithms (e.g.\ binary decision diagrams, minimal cut set expansions, or simulation-based techniques).  The unified \acrshort{pdag} structure codifies both the forward scenario expansions (\acrshort{et} logic) and the top-down sub-component dependencies (\acrshort{ft} logic) without creating contradictions or cycles. Definition~\ref{def:unified_pra_dag} makes manifest \emph{which} event tree references \emph{which} fault tree, how fault trees are chained together, and how basic events ultimately feed every higher-level node.  Once built, one may:
\begin{itemize}
\item \textbf{Traverse} each path from an initiating event to a final end-state, propelling forward along the \acrshort{et} edges and evaluating any \acrshort{ft} nodes via \(\pi_{F}\).
\item \textbf{Find minimal cut sets and path sets} by traversing the \acrshort{pdag} and making cuts along the way.
\item \textbf{Sum} (or bound, or approximate) scenario probabilities to quantify overall risk.  
\item \textbf{Perform sensitivity analyses} by biasing subsets of basic-event probabilities \(p(b)\) or gate dependencies.
\end{itemize}
All standard \acrshort{pra} methods (minimal cut sets, Monte Carlo simulation, bounding formulas, etc.) remain applicable, but now from within a single unified \acrshort{pdag} representation.

\section{Layered Topological Organization}
\label{sec:layered_dag_traversal}

Recall that a \acrshort{pdag} \(\mathcal{G} = (\mathcal{V}, \mathcal{E})\) contains no cycles, so there is at least one valid \emph{topological ordering} of its nodes.  A topological ordering assigns each node a numerical \emph{layer index} such that all edges point from a lower-numbered layer to a higher-numbered layer. If a node \(v\) consumes the outputs of nodes \(\{u_1,\dots,u_k\}\), then we require
\[
\text{layer}(u_i) \;<\; \text{layer}(v)
\quad
\text{for each }i\in\{1,\dots,k\}.
\]
In other words, node \(v\) can appear only after all of its inputs in a linear or layered listing.

The essential steps to build and traverse these layers are:

\begin{enumerate}
    \item \emph{Compute Depths via Recursive Analysis:}  
      Each node's depth is found by inspecting its children (or inputs).  If a node is a leaf (e.g., a \texttt{Variable} or \texttt{Constant} that does not depend on any other node), its depth is 0.  Otherwise, its depth is one larger than the maximum depth among its children.  

    \item \emph{Group Nodes by Layer:}  
      Once each node's depth is computed, nodes of equal depth form a single \emph{layer}. Thus, all nodes with depth \(0\) are in the first layer, those with depth \(1\) in the second layer, and so on.  

    \item \emph{Sort Nodes within Each Layer:}  
      Within each layer, enforce an additional consistent ordering: (i)~variables appear before gates, (ii)~gates of different types can be grouped to facilitate specialized processing.  This step is not strictly required for correctness, but it can streamline subsequent stages such as kernel generation or partial evaluations.

    \item \emph{Traverse Layer by Layer:}  
      A final pass iterates over each layer in ascending order.  Because all inputs of any node in layer \(d\) lie in layers \(< d\), the evaluation (or "kernel build") for layer \(d\) can proceed after the entire set of layers \(0,\dots,d-1\) is processed.
\end{enumerate}

This structure ensures a sound evaluation of the \acrshort{pdag}: no gate or variable is computed until after all of its inputs are finalized.

\subsection{Depth Computation and Node Collection}

\begin{enumerate}
    \item \textbf{Clear Previous State.}  
      Any existing "visit" markers or stored depths in the \acrshort{pdag}-based data structures are reset to default values (e.g., zero or -1).
      
    \item \textbf{Depth Assignment by Recursion.}  
      A \texttt{compute\_depth} subroutine inspects each node:
      \begin{enumerate}
        \item If the node is a \texttt{Variable} or \texttt{Constant}, it is a leaf in the \acrshort{pdag}, so depth \(=0\).  
        \item If the node is a \texttt{Gate} with multiple inputs, the procedure first recursively computes the depths of its inputs. It then sets its own depth as 
        \[
          \text{depth}(\texttt{gate})
          \;=\;
          1 \;+\;\max\limits_{\ell \in \text{inputs of gate}} \Bigl[\text{depth}(\ell)\Bigr].
        \]
      \end{enumerate}
    \item \textbf{Order Assignment.}  
      Each node stores the newly computed depth in an internal field. This numeric value anchors the node to a layer. A consistent pass over the entire graph ensures correctness for all nodes.
\end{enumerate}

After depths are assigned, gather all nodes, walking the \acrshort{pdag} from its root, recording each discovered node and adding it to a global list.

\subsection{Layer Grouping and Local Sorting}

Begin by creating:
\begin{itemize}
\item A global list of all nodes, each with a valid depth,  
\item A mapping from node indices to node pointers,  
\end{itemize}
Then, sort the global list by ascending depth.  Let \(\text{order}(n)\) be the depth of node \(n\).  Then
\[
\text{order}(n_1)\;\le\;\text{order}(n_2)\;\le\;\dots\,\le\;\text{order}(n_{|\mathcal{V}|}).
\]
Finally, partition this list into contiguous \emph{layers}: if the deepest node has a depth \(\delta_{\max}\), then create sub-lists:
\[
\{\text{nodes s.t. depth}=0\},
\quad
\{\text{nodes s.t. depth}=1\},
\quad
\dots,
\quad
\{\text{nodes s.t. depth}=\delta_{\max}\}.
\]
Within each layer, sort nodes to ensure that \texttt{Variable} nodes precede \texttt{Gate} nodes, and \texttt{Gate} nodes may be further sorted by \texttt{Connective} type (e.g., \texttt{AND}, \texttt{OR}, \texttt{VOT}, etc.).

\subsection{Layer-by-Layer Kernel Construction}

Apply the layer decomposition to drive \emph{kernel building} and \emph{evaluation}:

\begin{enumerate}
    \item \textbf{Iterate over each layer in ascending depth}.  Because every node's dependencies lie in a strictly lower layer, one is guaranteed that those dependencies have already been assigned memory buffers, partial results, or other necessary resources.
    \item \textbf{Partition the layer nodes into subsets by node type}.  Concretely, \texttt{Variable} nodes are batched together for \emph{basic-event sampling} kernels, while \texttt{Gate} nodes are transferred into \emph{gate-evaluation} kernels.  
    \item \textbf{Generate device kernels}.  For \texttt{Variable} nodes, create Monte Carlo sampling kernels. For \texttt{Gate} nodes, it constructs logical or bitwise operations that merge or transform the sampled states of the inputs.  
\end{enumerate}

Once kernels for a given layer finish, move on to the next layer. Because of the topological guarantee, no node in layer \(d\) references memory or intermediate states from layer \(d\!+\!1\) or later, preventing cyclical references and ensuring correctness.
\section{Bitpacked Random Number Generator}

Monte Carlo simulations, probability evaluations, and other sampling-based procedures benefit greatly from efficient, high-quality \acrfull{rng}s. A large class of modern \acrshort{rng}s are known as \textit{counter-based \acrfull{prng}s}, because they use integer counters (e.g., 32-bit or 64-bit) along with a stateless transformation to produce random outputs. The \emph{Philox} family of counter-based \acrshort{prng}s is a well-known example, featuring fast generation, high period, and good statistical properties. In this section, we discuss the general principles of counter-based \acrshort{prng}s, explain how Philox fits into this paradigm, analyze its complexity, and present a concise pseudocode version of the \(\text{Philox }4\times32\text{-10}\) variant. Subsequently, we detail the bitpacking scheme used to reduce memory consumption when storing large numbers of Bernoulli samples.

A counter-based \acrshort{prng} maps a user-supplied \emph{counter} (plus, optionally, a \emph{key}) to a fixed-size block of random bits via a deterministic function. Formally, if 
\[
  \mathbf{x} \;=\; (x_1, x_2, \ldots, x_k)
\]
is a vector of one or more 32-bit or 64-bit counters, and 
\[
  \mathbf{k} \;=\; (k_1, k_2, \ldots, k_m)
\]
is a key vector, then a counter-based \acrshort{prng} defines a transformation 
\[
   \mathcal{F}(\mathbf{x}, \mathbf{k})
   \;=\;
   (\rho_1, \rho_2, \ldots, \rho_r),
\]
where each \(\rho_j\) is typically a 32-bit or 64-bit output. Different increments of the counter \(\mathbf{x}\) produce different pseudo-random outputs \(\rho_j\). The process is stateless in the sense that advancing the RNG amounts to incrementing the counter (e.g., \(\mathbf{x}\mapsto \mathbf{x} + 1\)).

Compared to older recurrence-based \acrshort{rng}s such as linear congruential generators or the Mersenne Twister, counter-based methods offer more straightforward parallelization, reproducibility across multiple streams, and strong structural simplicity: no internal state must be updated or maintained. This is particularly valuable in distributed Monte Carlo simulations or \acrshort{gpu}-based sampling, where each thread or work-item can be assigned a different counter. Philox constructs its pseudo-random outputs by applying a small set of mixed arithmetic (multiplication/bitwise) rounds to an input \emph{counter} plus \emph{key}. In particular, \(\mathrm{Philox}\,4\times32\text{-10}\) (often shortened to “Philox-4x32-10”) works on four 32-bit integers at a time:
\[
  \mathbf{S} = (S_0, S_1, S_2, S_3),
  \qquad
  \mathbf{K} = (K_0, K_1).
\]
The four elements \(\{S_0, S_1, S_2, S_3\}\) collectively represent the counter, e.g., \((x_0, x_1, x_2, x_3)\). The two key elements \((K_0, K_1)\) are used to tweak the generator’s sequence. A single invocation of Philox-4x32-10 transforms \(\mathbf{S}\) into four new 32-bit outputs after ten rounds of mixing. At each round, the algorithm:
\begin{enumerate}
    \item Multiplies two of the state words by fixed “magic constants” to create partial products.
    \item Takes the high and low 32-bit portions of those 64-bit products.
    \item Incorporates the round key to shuffle the words.
    \item Bumps the key by adding constant increments \((\mathrm{W32A} = 0x9E3779B9 \text{ and } \mathrm{W32B} = 0xBB67AE85)\).
\end{enumerate}
After ten rounds, the final \((S_0, S_1, S_2, S_3)\) is returned as the pseudo-random block. A new call to Philox increases the counter \(\mathbf{S}\) by one (e.g., \(S_3 \mapsto S_3 + 1\)) and re-enters the same function. The Philox-4x32-10 algorithm is designed so that each blocking call requires a \emph{constant number} of operations, independent of the size of any prior “state.” Specifically, each round involves:
\[
  \mathcal{O}(1)\;\text{ arithmetic operations},
\]
and there are \(\mathrm{R} = 10\) rounds. Thus, each Philox invocation is asymptotically constant time \(\mathcal{O}(\mathrm{R}) = \mathcal{O}(1)\). The total cost to generate 128 bits (4 words \(\times\) 32 bits) is therefore constant time per call.

\subsection{The 10-round Philox-4x32}
Our implementation follows the standard 10-round approach for generating one block of four 32-bit random words, also called Philox-4x32-10. Let \(M_{\mathrm{A}}=0xD2511F53\), \(M_{\mathrm{B}}=0xCD9E8D57\) be the multipliers, and let \((K_0, K_1)\) be the key which is updated each round by \(\mathrm{W32A}=0x9E3779B9\) and \(\mathrm{W32B}=0xBB67AE85\). The function \(\text{Hi}(\cdot)\) returns the high 32 bits of a 64-bit product, and \(\text{Lo}(\cdot)\) returns the low 32 bits. Because each call produces four 32-bit pseudo-random words, Philox-4x32-10 is particularly convenient for batched sampling. If only a single 32-bit word is needed, one can still call the function and discard the excess words; however, many applications consume all four outputs (e.g., to produce four floating-point variates).

\begin{algorithm}[ht]
  \caption{Philox‑4x32‑10}\label{alg:philox}
  \begin{algorithmic}[1]
    %------------------------------------------------------------
    \Require Four 32‑bit counters $(S_0,S_1,S_2,S_3)$,
            key $(K_0,K_1)$
    \Ensure  Transformed counters $(S_0,S_1,S_2,S_3)$
    %------------------------------------------------------------
    \Statex
    %--------------------- Philox_Round -------------------------
    \Procedure{Philox\_Round}{$(S_0,S_1,S_2,S_3),\,(K_0,K_1)$}
      \State $P_0 \gets M_{\text{A}}\times S_0$ \Comment{64‑bit product}
      \State $P_1 \gets M_{\text{B}}\times S_2$ \Comment{64‑bit product}
      \State $T_0 \gets \mathrm{Hi}(P_1)\,\oplus\,S_1\,\oplus\,K_0$
      \State $T_1 \gets \mathrm{Lo}(P_1)$
      \State $T_2 \gets \mathrm{Hi}(P_0)\,\oplus\,S_3\,\oplus\,K_1$
      \State $T_3 \gets \mathrm{Lo}(P_0)$
      \State $K_0 \gets K_0 + \mathrm{W32A}$
      \State $K_1 \gets K_1 + \mathrm{W32B}$
      \State \Return $\bigl((T_0,T_1,T_2,T_3),\,(K_0,K_1)\bigr)$
    \EndProcedure
    %------------------- Philox4x32_10 --------------------------
    \Statex
    \Procedure{Philox4x32\_10}{$(S_0,S_1,S_2,S_3),\,(K_0,K_1)$}
      \For{$i \gets 1$ \textbf{to} 10}
        \State $\bigl(S_0,S_1,S_2,S_3),\,(K_0,K_1) \gets$
               \Call{Philox\_Round}{$(S_0,S_1,S_2,S_3),\,(K_0,K_1)$}
      \EndFor
      \State \Return $(S_0,S_1,S_2,S_3)$
    \EndProcedure
  \end{algorithmic}
\end{algorithm}

\subsection{Bitpacking for Probability Sampling}
It takes exactly one bit to represent the outcome of a trial. If these If outcomes are stored naively, each one occupies a full 8-bit byte. Hence, only \( \tfrac{1}{8} \) of the allocated space is used for actual data. By instead packing up to \(w\) indicators into a \(w\)-bit machine word, the memory usage can be reduced by a factor of up to \(8\) (in the simplest scenario of 8-bit groupings). In more general terms:
\[
  \text{Memory usage }M_{\text{naive}}
  \;=\;
  N \times 8\;\text{bits},
  \qquad
  \text{Memory usage }M_{\text{pack}}
  \;=\;
  \left\lceil\frac{N}{w}\right\rceil \,\times\,w\;\text{bits}.
\]
In our implementation, each call to Philox-4x32-10 yields 128 bits of randomness. We use those bits to draw exactly 128 Bernoulli outcomes at once, then combine them into a \(\mathrm{bitpack}\) of two 64-bit integers. For instance, if we choose a batch size of \(4\)-bits to represent four Bernoulli samples in a single chunk, we can:

\begin{enumerate}
    \item Generate a block \(\{r_0, r_1, r_2, r_3\}\) of four 32-bit random integers from Philox.
    \item Convert each \(r_i\) into a uniform \([0,1)\) floating-point value by dividing by \(2^{32}\).
    \item Compare each to the target probability \(p\).
    \item Form a 4-bit integer, each bit set to \(1\) if the corresponding comparison succeeded, or \(0\) otherwise.
\end{enumerate}

Repeating these steps for multiple rounds of 4 bits each can fill a 16-bit or 32-bit \(\mathrm{bitpack}\) variable with many Bernoulli indicators. Then it can be stored into an array at a single index, reducing memory overhead by constant factor of $N$. 

\begin{algorithm}[H]
  \caption{Bit‑packing of four Bernoulli samples into a 4‑bit block}
  \label{alg:four_bit_pack}
  \begin{algorithmic}[1]
    %------------------------------------------------------------
    \Require Probability $p\in[0,1]$, random 32‑bit words $(r_0,r_1,r_2,r_3)$
    \Ensure  4‑bit integer bits containing the four Bernoulli draws
    %------------------------------------------------------------
    \Procedure{FourBitPack}{$p,(r_0,r_1,r_2,r_3)$}
      \State bits $\gets 0$
      \For{$i \gets 0$ \textbf{to} 3}
        \State $u_i \gets r_i / 2^{32}$ \Comment{$u_i\in[0,1)$}
        \If{$u_i < p$}
            \State $b_i \gets 1$
        \Else
            \State $b_i \gets 0$
        \EndIf
        \State bits $\gets$ bits $\mid (b_i \ll i)$ 
               \Comment{set bit $i$ to $b_i$}
      \EndFor
      \State \Return bits
    \EndProcedure
  \end{algorithmic}
\end{algorithm}

In this procedure, \(\Vert\) denotes a bitwise OR, and \(\ll\) denotes a left shift. One then repeats the above call to accumulate multiple 4-bit blocks (e.g., for a total of 16 bits, one calls FourBitPack four times and merges the results with the appropriate shifts).
\section{Tallying Layer Outputs}
\label{sec:tally_kernel}

At every Monte-Carlo iteration the simulator produces, for each logic node
\(v\in \mathcal{V}\), a bit-packed buffer encoding
\[
  \mathbf{Y}_v^{(t)}
  \;=\;
  \bigl(y_{v,1}^{(t)}, y_{v,2}^{(t)},\dots, y_{v,N}^{(t)}\bigr)
  \in\{0,1\}^N,
  \quad t = 1,\dots,T,
\]
where \(N\!=\!B\!\times\!P\!\times\!\omega\) is the number of Bernoulli trials
per Monte-Carlo \emph{iteration}:
\begin{itemize}
    \item \(B\) - number of \emph{batches},
    \item \(P\) - bit-packs per batch,
    \item \(\omega\!=\!8\cdot\mathrm{sizeof}(\text{bitpack\_t})\) - bits per pack.
\end{itemize}
Because the buffers are overwritten at the next iteration, a
separate \emph{tally layer} accumulates summary statistics that persist for the
entire simulation.  The present section formalises that process and outlines
an implementation-agnostic, data-parallel algorithm that realises it on modern
accelerators.

\subsection{Statistical objectives}
\label{subsec:tally_objective}

For every node \(v\) we wish to estimate, after \(T\) Monte-Carlo iterations,

\[
  \widehat{p}_v
  \;=\;
  \frac{1}{T\,N}
  \sum_{t=1}^{T}\sum_{j=1}^{N} y_{v,j}^{(t)}
  \;=\;
  \frac{s_v}{T\,N},
  \qquad
  s_v \;=\; \text{total \# of one-bits observed for node \(v\)}.
\]

Under the usual independence assumptions the sampling distribution of
\(\widehat{p}_v\) is asymptotically
\(\mathcal{N}\!\bigl(p_v,\,
  \tfrac{p_v(1-p_v)}{T\,N}\bigr)\).
Hence

\[
  \widehat{\sigma}_v
  \;=\;
  \sqrt{\frac{\widehat{p}_v\,(1-\widehat{p}_v)}{T\,N}}
\]

is an unbiased estimator of the standard error, giving the
\((1-\alpha)\)\,--\,level normal confidence interval

\[
  \bigl[
    \widehat{p}_v - z_{1-\alpha/2}\,\widehat{\sigma}_v,\;
    \widehat{p}_v + z_{1-\alpha/2}\,\widehat{\sigma}_v
  \bigr],
  \qquad
  z_{1-\alpha/2}\in\{1.96,\,2.58,\dots\}.
\]

The tally routine therefore needs to maintain only the scalar
\(s_v\) while the simulation is running; the derived statistics can be updated
in-place whenever a user requests intermediate results or at a fixed cadence.

\subsection{Parallel accumulation algorithm}

The accumulation kernel is invoked on a three-dimensional
\texttt{nd\_range}, chosen such that
\[
  \begin{aligned}
    \text{global}_x &\;\ge\; V,\\
    \text{global}_y &\;\ge\; B,\\
    \text{global}_z &\;\ge\; P.
  \end{aligned}
\]
Work-item \((i_x,i_y,i_z)\) is responsible for \emph{exactly one} bit-pack:
\[
  \text{node  } v=i_x,\quad
  \text{batch } b=i_y,\quad
  \text{pack  } p=i_z.
\]

\vspace{4pt}
\noindent
\textbf{Local workflow of a work-item}
\begin{enumerate}
    \item Load the \(p^{\text{th}}\) bit-pack of batch \(b\) from
          \texttt{buffer}.
    \item Compute \(c=\mathrm{popcount}(\text{bitpack})\).
    \item Reduce the \(c\)'s belonging to the same work-\emph{group} in
          shared memory (tree reduction or \texttt{reduce\_over\_group}).
    \item One designated leader performs
          \(\texttt{atomic\_add}(\texttt{num\_one\_bits},\,\text{group\_sum})\).
\end{enumerate}

The reduction ensures only one atomic operation per group, greatly reducing
contention when \(P\) is large.

We present platform-neutral pseudocode that encapsulates the above logic while remaining agnostic to the underlying API.

\paragraph{Host-side control.}
After each Monte-Carlo iteration the host enqueues \textsc{TallyKernel} with a
fresh \texttt{iteration} counter.  When either (i)~a user requests
intermediate statistics or (ii)~a pre-set reporting interval is reached,
the host reads back \texttt{num\_one\_bits} and executes the purely
serial routine shown in Algorithm~\ref{alg:update_stats}.

\begin{algorithm}[H]
\caption{Post-processing of a single node's tally}
\label{alg:update_stats}
\begin{algorithmic}[1]
  \Require
    \(s\) - total one-bits,
    \(T\), \(B\), \(P\), \(\omega\) - run parameters
  \Ensure
    \(\widehat{p}\), \(\widehat{\sigma}\), two symmetric CIs
  \State $N\gets B\cdot P\cdot\omega$
  \State $\widehat{p}\gets s / (T\,N)$
  \State $\widehat{\sigma}\gets
          \sqrt{\widehat{p}(1-\widehat{p})/(T\,N)}$
  \For{\textbf{each} $z\in\{1.96,\,2.58\}$}
      \State $\text{CI}\gets
        \bigl[\max(0,\widehat{p}-z\widehat{\sigma}),
              \min(1,\widehat{p}+z\widehat{\sigma})\bigr]$
  \EndFor
\end{algorithmic}
\end{algorithm}

The above normal approximation is valid provided \(T\,N\widehat{p}\)
and \(T\,N(1-\widehat{p})\) both exceed roughly 10; otherwise an exact
Clopper-Pearson interval can be substituted with no change to the running
sum logic.

\subsection{Correctness and complexity}

\textbf{Work-item cost.}
Each work-item performs one \(\mathrm{popcount}\) and
participates in an \(O(\log L)\) intra-group reduction
(\(L\!=\!\text{local\_range}\)), yielding an overall
\(O(\log L)\) instruction count.

\textbf{Global cost.}
The total number of work-items launched per iteration is
\(V\cdot B\cdot P\).  Because each bit-pack contains \(\omega\) Bernoulli
trials, the cost \emph{per trial} shrinks as \(\omega^{-1}\).

\textbf{Memory traffic.}
Every work-item reads exactly one machine word and no writes occur except
the single atomic addition per work-group.  Hence the algorithm is
memory-bandwidth bound only at extremely low arithmetic intensity
(\(P\approx 1\)).

\textbf{Linear scalability.}
All tally nodes are independent.  Increasing \(V\) therefore scales the total
runtime linearly until either (i)~the device saturates its occupancy or
(ii)~atomic contention becomes non-negligible; the group-level reduction
mitigates the latter.

\section{Preliminary Benchmarks on Arialia Fault Trees}
\subsection{Runtime Environment and Benchmarking Setup}
\label{subsec:runtime_environment}

All experiments were performed on a consumer-grade desktop provisioned with an NVIDIA\textsuperscript{\textregistered} GeForce GTX~1660~SUPER graphics card (1{,}408~CUDA cores, 6\,\acrshort{gb} of dedicated GDDR6 memory) and a 10th-generation Intel\textsuperscript{\textregistered} Core\textsuperscript{TM}~i7-10700 \acrshort{cpu} (2.90\,GHz base clock, with turbo-boost and hyperthreading enabled). The code implementation relies on \acrshort{sycl} using the AdaptiveCpp (formerly HipSYCL) framework, which employs an LLVM based runtime and \acrfull{jit} kernel compilation.

\subsubsection*{Monte Carlo Sampling Strategy}
Each fault tree model was evaluated through a single pass (one iteration), generating as many Monte Carlo samples as would fit into the \acrshort{gpu}’s 6\,\acrshort{gb} memory. A 64-bit counter-based Philox4x32x10 random number generator was applied in parallel to produce the basic-event realizations, ensuring repeatability and independence for large sample counts.

\subsubsection*{Bit-Packing and Data Types}
To reduce memory usage and increase vectorized throughput, every batch of Monte Carlo results was bit-packed into 64-bit words. Accumulated tallies of successes or failures were stored as 64-bit integers, while floating-point calculations (e.g., probability estimates) used double precision (64-bit floats). These design decisions are intended to maintain numerical consistency and make use of native hardware operations (such as population-count instructions for threshold gates).

\subsubsection*{Execution Procedure}
Upon launching the application, the enabling overhead (host-device transfers, \acrshort{jit} compilation, and kernel configuration) was included in the total wall-clock measurement. Each benchmark was compiled at the \texttt{-O3} optimization level to ensure efficient instruction generation. Every experiment was repeated at least five times, and measured runtimes were averaged to reduce the impact of transient background processes or scheduling variations on the host system.

\subsection{Assumptions and Constraints}
The primary objective was to gauge runtime across a set of fault trees that vary widely in size, logic complexity, and probability ranges within a typical Monte Carlo integration workflow. The experiments assume independent operation of the test machine, with no significant other processes contending for \acrshort{gpu} or \acrshort{cpu} resources. All sampling took place within a single pass, so the measured wall times incorporate initial kernel launches, memory copies, and statistical collection of gate outcomes. No specialized forms of hardware optimization beyond the data-parallel approach (e.g., pinned memory or asynchronous streams) were used.

\subsection{Comparative Accuracy \& Runtime}

Table~\ref{tab:canopy-logp-mae} and Figure~\ref{fig:canopy_rel_error_plot} summarize the accuracy of three approximate quantification methods \acrfull{rea}, \acrfull{mcub}, and our \acrshort{gpu}-accelerated Monte Carlo by listing each approach's mean relative error in the log-probability (\(\log p\)) domain, alongside the total MC samples and runtime. Although each fault tree exhibits its own complexities, several broad trends emerge:

\begin{enumerate}
    \item \textbf{\acrshort{rea} accuracy strongly depends on the \emph{actual} top-event probability.}
    \begin{itemize}
        \item For trees with very low-probability failures (e.g., \texttt{baobab1}, \texttt{das9202}, \texttt{isp9605}), where individual component failures rarely coincide, \acrshort{rea}'s mean error often remains near or below \(10^{-2}\) in log space. This indicates that summing only the first-order minimal cut sets--assuming higher-order intersections contribute negligibly--can be valid when the system is indeed dominated by single-component or few-component events.
        \item However, for fault trees with moderate or higher top-event probabilities (\(\gtrsim 10^{-2}\)), \acrshort{rea}'s inaccuracy tends to grow (for instance, up to \(10^{-1}\) in \texttt{edf9203}, \texttt{edf9204}, and \texttt{edfpa15b}). In these cases, ignoring the overlap of multiple cut sets leads to a visible systematic error.
    \end{itemize}

    \item \textbf{Min-Cut Upper Bound (\acrshort{mcub}) often mirrors \acrshort{rea} but with exaggerated errors in certain overlapping cut configurations.}
    \begin{itemize}
        \item In many models (e.g., \texttt{cea9601}, \texttt{baobab3}, \texttt{das9601}), \acrshort{mcub} closely tracks \acrshort{rea}, suggesting that higher-order combinations remain negligible in those systems.
        \item Yet, in a few cases involving heavy cut-set overlap (e.g., \texttt{das9209}, row~14), \acrshort{mcub} soars to a mean log-probability error of \(\sim 17\), dwarfing \acrshort{rea} or Monte Carlo. This highlights the well-known pitfall: if multiple cut sets are not genuinely ``rare'' and substantially overlap, the union bound becomes extremely loose.
    \end{itemize}

    \item \textbf{Monte Carlo yields more consistent and often dramatically lower numerical errors for most moderate- to high-probability top events.}
    \begin{itemize}
        \item For example, in \texttt{das9201} (row~6) and \texttt{edf9203} (row~19), the Monte Carlo error is well below \(10^{-3}\), whereas both \acrshort{rea} and \acrshort{mcub} can exceed \(10^{-1}\). In these situations, ignoring or bounding higher-order intersections proves inadequate, while direct sampling naturally captures all overlaps.
        \item However, for fault trees with extremely small top-event probabilities, Monte Carlo's variance can become harder to control. For instance, some rows (\texttt{das9204}, \texttt{das9205}, \texttt{isp9605}, \texttt{isp9607}) show that roughly \(10^{8}\)--\(10^{9}\) samples are required to constrain the error within a few tenths in \(\log p\). Those entries either exhibit a slightly higher Monte Carlo error than \acrshort{rea}/\acrshort{mcub} or demonstrate that we needed a disproportionately large sample count (and thus more runtime) to compete with simple rare-event approximations.
    \end{itemize}

    \item \textbf{Sampling scale and runtime remain surprisingly feasible, even for up to \(10^{9}\) draws.}
    \begin{itemize}
        \item Despite some test cases sampling in the hundreds of millions or billions, runtimes remain \(\approx 0.2\)--\(0.3\)~s for most fault trees, rarely exceeding 1~s (see, for instance, row~10 with 3.3~B samples and \(\sim 0.96\)~s). This indicates that the bit-packed, data-parallel Monte Carlo engine is highly optimized, making large-sample simulation a viable alternative to purely analytical approaches for many real-world PRA problems.
        \item By contrast, the bounding methods (\acrshort{rea} and \acrshort{mcub}) typically run in negligible time but deliver inconsistent accuracy depending on each tree's structure. In practice, a hybrid strategy may emerge: apply bounding methods for quick estimates, then selectively invoke large-sample Monte Carlo for trees or subsections where the bounding approximation diverges.
    \end{itemize}

    \item \textbf{Omitted or Extreme Cases.}
    \begin{itemize}
        \item Rows where Monte Carlo entries are missing (e.g., \texttt{das9209} and \texttt{edf9206}) indicate difficulty in converging to a useful estimate within a fixed iteration budget. Conversely, \acrshort{mcub} shows erratic jumps in some of those same cases, underlining the fact that both bounding and sampling approaches can struggle in certain outliers.
        \item Model \texttt{nus9601} (row~43) lacks all three error columns since no reference solution was available, reflecting a scenario where direct verification remains pending or inapplicable. Nevertheless, the completion time of \(\sim 0.29\)~s for a partial exploration suggests that the structural overhead of large fault trees can still be handled efficiently.
    \end{itemize}
\end{enumerate}

These results affirm that Monte Carlo methods, when equipped with high throughput sampling, can achieve the most robust accuracy across a broader spectrum of top-event probabilities, particularly in configurations where standard cut set approximations fail to capture significant event dependencies. At the same time, rare-event with exceptionally small probabilities can pose challenges for naive sampling, revealing the potential need for adaptive variance-reduction techniques or partial enumerations. In practice, analysts may combine bounding calculations (\acrshort{rea}/\acrshort{mcub}) for quick screening or preparatory checks, then use hardware-accelerated Monte Carlo to refine those domains most susceptible to underestimation or overestimation by simpler approximations. Alternatively, for very large models, where exact solutions may be unavailable, data-parallel Monte Carlo can still estimate event probabilities without building minimal cut sets. 

% \begin{landscape}
\sisetup{ 
  scientific-notation = true,
  table-format        = 1.2e-2,
  round-mode          = places,
  round-direction     = up,
  round-precision     = 2,
  group-separator     = {,},
  group-minimum-digits= 4,
  table-text-alignment=center,
  table-number-alignment = center,
}
\begin{longtable}{@{}l
                     l
                     S
                     S
                     S
                     S[table-format=1.1e-2,round-precision=1]
                     S[scientific-notation=false, table-format=1.3]@{}}
\caption{Relative error (Log-probability), \acrfull{dpmc} vs \acrfull{mcub} and \acrfull{rea}.}
\label{tab:canopy-logp-mae}\\
\toprule
\textbf{\#} &
  \makecell{\textbf{Fault}\\\textbf{Tree}} &
  \multicolumn{3}{c}{\textbf{Relative Error $\left| \log_{10}\left(\frac{P_{\text{approx}}}{P_{\text{exact}}}\right) \right|$}} &
  \textbf{Samples} &
  \textbf{Runtime (\si{\second})} \\
\cmidrule(lr){3-5}
& & \textbf{\acrshort{rea}} & \textbf{\acrshort{mcub}} & \textbf{\acrshort{dpmc}} & & \\
\midrule
\endfirsthead
\multicolumn{7}{c}{\textit{Continued: Relative Error (Log-Probability), \acrshort{dpmc} vs \acrshort{mcub}, \acrshort{rea}.}}\\
\toprule
\textbf{\#} &
  \makecell{\textbf{Fault}\\\textbf{Tree}} &
  \multicolumn{3}{c}{\textbf{Relative Error $\left| \log_{10}\left(\frac{P_{\text{approx}}}{P_{\text{exact}}}\right) \right|$}} &
  \textbf{Samples} &
  \textbf{Runtime (\si{\second})} \\
\cmidrule(lr){3-5}
& & \textbf{\acrshort{rea}} & \textbf{\acrshort{mcub}} & \textbf{\acrshort{dpmc}} & & \\
\midrule
\endhead
\bottomrule
\endfoot
%
\endlastfoot
%
\textbf{1}  & baobab1  & 1.45156E-04 & 1.45156E-04 & 7.61880E-03                         & 2.5E+08 & 0.262 \\
\textbf{2}  & baobab2  & 6.48628E-03 & 6.34705E-03 & 1.54436E-03 & 2.5E+08 & 0.209 \\
\textbf{3}  & baobab3  & 1.21509E-02 & 1.16701E-02 & 2.24843E-04 & 2.4E+08 & 0.259 \\
\textbf{4}  & cea9601  & 9.36195E-02 & 9.32207E-02 & 2.41802E-03 & 1.2E+08 & 0.262 \\
\textbf{5}  & chinese  & 1.08742E-02 & 1.06354E-02 & 2.14601E-03 & 9.4E+08 & 0.277 \\
\textbf{6}  & das9201  & 1.26649E-01 & 1.22765E-01 & 5.49963E-05 & 2.3E+08 & 0.279 \\
\textbf{7}  & das9202  & 7.72743E-05 & 2.57596E-05 & 1.20232E-04                         & 5.2E+08 & 0.295 \\
\textbf{8}  & das9203  & 3.59019E-02 & 3.55935E-02 & 2.31768E-04 & 5.2E+08 & 0.292 \\
\textbf{9}  & das9204  & 1.68086E-01 & 1.68087E-01 & 1.13495E-01 & 6.1E+08 & 0.292 \\
\textbf{10} & das9205  & 9.63825E-02 & 9.63725E-02 & 2.76190E-02 & 3.3E+09 & 0.958 \\
\textbf{11} & das9206  & 5.43561E-02 & 8.89660E-04 & 3.51548E-04 & 2.0E+08 & 0.269 \\
\textbf{12} & das9207  & 1.18486E-01 & 2.45492E-02 & 1.36519E-04 & 9.5E+07 & 0.282 \\
\textbf{13} & das9208  & 4.12808E-02 & 3.81968E-02 & 9.34017E-05 & 2.5E+08 & 0.307 \\
\textbf{14} &
  das9209 &
  2.11242E-02 &
  1.70245E+01 &
   &
   &
   \\
\textbf{15} & das9601  & 5.29285E-02 & 5.19122E-02 & 6.67174E-04 & 1.1E+08 & 0.256 \\
\textbf{16} & das9701  & 5.02804E-02 & 3.37565E-02 & 6.22978E-04 & 2.3E+07 & 0.273 \\
\textbf{17} & edf9201  & 1.48012E-01 & 5.36182E-02 & 2.88906E-04 & 1.8E+08 & 0.315 \\
\textbf{18} & edf9202  & 1.07181E-01 & 6.05976E-03 & 4.53900E-04 & 7.8E+07 & 0.271 \\
\textbf{19} & edf9203  & 2.22146E-01 & 1.17293E-01 & 3.27993E-04 & 8.0E+07 & 0.302 \\
\textbf{20} & edf9204  & 2.79531E-01 & 1.05591E-01 & 1.31416E-04 & 8.7E+07 & 0.298 \\
\textbf{21} & edf9205  & 9.94339E-02 & 4.46260E-02 & 5.60146E-05 & 1.9E+08 & 0.284 \\
\textbf{22} & edf9206  & 6.98797E-03 & 7.07775E-03 &                                    &        &      \\
\textbf{23} & edfpa14b & 1.85574E-01 & 9.15983E-02 & 1.04767E-03 & 9.4E+07 & 0.267 \\
\textbf{24} & edfpa14o & 1.86482E-01 & 9.18665E-02 & 3.39049E-04 & 9.8E+07 & 0.275 \\
\textbf{25} & edfpa14p & 3.40010E-02 & 1.66283E-02 & 5.35099E-04 & 2.1E+08 & 0.294 \\
\textbf{26} & edfpa14q & 1.85609E-01 & 9.15366E-02 & 3.33292E-04 & 9.6E+07 & 0.282 \\
\textbf{27} & edfpa14r & 2.48088E-02 & 2.09729E-02 & 9.33865E-04 & 2.1E+08 & 0.294 \\
\textbf{28} & edfpa15b & 2.16329E-01 & 9.37065E-02 & 4.67881E-04 & 1.1E+08 & 0.283 \\
\textbf{29} & edfpa15o & 2.16502E-01 & 9.37627E-02 & 4.06846E-05 & 1.1E+08 & 0.282 \\
\textbf{30} & edfpa15p & 2.52568E-02 & 1.00382E-02 & 3.54344E-04 & 2.6E+08 & 0.299 \\
\textbf{31} & edfpa15q & 2.16329E-01 & 9.37065E-02 & 6.74736E-04 & 1.1E+08 & 0.284 \\
\textbf{32} & edfpa15r & 1.94693E-02 & 1.62668E-02 & 4.04924E-04 & 2.5E+08 & 0.290 \\
\textbf{33} & elf9601  & 1.98107E-02 & 8.08925E-05 & 7.86600E-05 & 2.3E+08 & 0.274 \\
\textbf{34} & ftr10    & 1.22076E-01 & 9.27268E-04 & 1.54844E-04 & 2.1E+08 & 0.297 \\
\textbf{35} & isp9601  & 8.08392E-02 & 6.63074E-02 & 1.13264E-04 & 1.8E+08 & 0.271 \\
\textbf{36} & isp9602  & 1.74572E-02 & 1.47782E-02 & 1.35280E-03 & 2.3E+08 & 0.281 \\
\textbf{37} & isp9603  & 3.82337E-02 & 3.74815E-02 & 3.82344E-03 & 2.7E+08 & 0.278 \\
\textbf{38} & isp9604  & 1.20889E-01 & 8.14313E-02 & 1.88665E-04 & 1.4E+08 & 0.280 \\
\textbf{39} & isp9605  & 6.57344E-03 & 6.57032E-03 & 2.93472E-02                         & 5.0E+08 & 0.262 \\
\textbf{40} & isp9606  & 2.27811E-02 & 1.18983E-02 & 1.30307E-04 & 3.4E+08 & 0.289 \\
\textbf{41} & isp9607  & 2.38880E-02 & 2.38880E-02 & 1.28136E-01                         & 3.8E+08 & 0.282 \\
\textbf{42} & jbd9601  & 1.22001E-01 & 1.35343E-02 & 1.08116E-04 & 5.7E+07 & 0.279 \\
\textbf{43} & nus9601  &            &            &                                    & 1.6E+07 & 0.289 \\* \bottomrule
\end{longtable}
% \end{landscape}


\subsection{Memory Consumption}

As mentioned previously, the memory was set to the maximum allocatable 6\acrshort{gb}, constrained by the NVIDIA GTX 1660 SUPER \acrshort{gpu}'s \acrshort{vram}. Figure \ref{fig:canopy_rel_error_plot_1} plots the actual consumed memory, as a function of \acrshort{pdag} input size and total number of bits sampled per node (gate or basic-event) per pass. Since there are multiple types of preprocessing steps, all of which affect the final size of the pruned \acrshort{pdag}, those have been plotted in Figure \ref{fig:canopy_rel_error_plot_2} for completeness. Since the nature of the actual pruning logic is not being benchmarked here, we named these v1, v2, v3 respectively. The key takeaways are that while some trees are more compressible than others, nearly all computations were performed by saturating available \acrshort{vram}. As a zoomed out version of Figure \ref{fig:canopy_rel_error_plot_1} , Figure \ref{fig:sampled_bits_mem} projects trends for the sampled bits count, as a function of model size, for varying amounts of available \acrshort{ram}.
\begin{landscape}
\begin{figure}[p]
    \centering
    \includesvg[width=1.38\textheight]{task_III/canopy/plots/rel_error.svg}
    \caption{Relative error (Log-probability) for \acrfull{dpmc} vs \acrfull{mcub} and \acrfull{rea}}
    \label{fig:canopy_rel_error_plot}
\end{figure}
\end{landscape}


\subsection{Aralia Fault Trees: Memory Consumption}


\begin{landscape}
\begin{figure}[h!]
    \centering
    \begin{subfigure}[t]{0.664\textwidth}
        \centering
        \includesvg[width=\linewidth]{task_III/canopy/plots/mem_allocation_lines_zoom.svg}
        \caption{Sampled Bits per Event per Iteration}
        \label{fig:canopy_rel_error_plot_1}
    \end{subfigure}
    \hfill
    \begin{subfigure}[t]{0.66\textwidth}
        \centering
        \includesvg[width=\linewidth]{task_III/canopy/plots/mem_allocation_lines_zoom_all.svg}
        \caption{Sampled Bits per Event per Iteration}
        \label{fig:canopy_rel_error_plot_2}
    \end{subfigure}
    \caption{Comparison of sampled bits per event per iteration.}
\end{figure}
\end{landscape}

\begin{figure}[h!]
    % \centerline{}
    \includesvg[width=0.95\textwidth]{task_III/canopy/plots/mem_allocation_lines_lg.svg}
    \caption{Sampled Bits per Event per Iteration}
    \label{fig: canopy_rel_error_plot}
\end{figure}
\input{4_proposed_solution/mc_solver/benchmarks/limitations}