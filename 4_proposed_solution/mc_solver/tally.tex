\section{Tallying Layer Outputs}
\label{sec:tally_kernel}

At every Monte-Carlo iteration the simulator produces, for each logic node
\(v\in \mathcal{V}\), a bit-packed buffer encoding
\[
  \mathbf{Y}_v^{(t)}
  \;=\;
  \bigl(y_{v,1}^{(t)}, y_{v,2}^{(t)},\dots, y_{v,N}^{(t)}\bigr)
  \in\{0,1\}^N,
  \quad t = 1,\dots,T,
\]
where \(N\!=\!B\!\times\!P\!\times\!\omega\) is the number of Bernoulli trials
per Monte-Carlo \emph{iteration}:
\begin{itemize}
    \item \(B\) - number of \emph{batches},
    \item \(P\) - bit-packs per batch,
    \item \(\omega\!=\!8\cdot\mathrm{sizeof}(\text{bitpack\_t})\) - bits per pack.
\end{itemize}
Because the buffers are overwritten at the next iteration, a
separate \emph{tally layer} accumulates summary statistics that persist for the
entire simulation.  The present section formalises that process and outlines
an implementation-agnostic, data-parallel algorithm that realises it on modern
accelerators.

\subsection{Statistical objectives}
\label{subsec:tally_objective}

For every node \(v\) we wish to estimate, after \(T\) Monte-Carlo iterations,

\[
  \widehat{p}_v
  \;=\;
  \frac{1}{T\,N}
  \sum_{t=1}^{T}\sum_{j=1}^{N} y_{v,j}^{(t)}
  \;=\;
  \frac{s_v}{T\,N},
  \qquad
  s_v \;=\; \text{total \# of one-bits observed for node \(v\)}.
\]

Under the usual independence assumptions the sampling distribution of
\(\widehat{p}_v\) is asymptotically
\(\mathcal{N}\!\bigl(p_v,\,
  \tfrac{p_v(1-p_v)}{T\,N}\bigr)\).
Hence

\[
  \widehat{\sigma}_v
  \;=\;
  \sqrt{\frac{\widehat{p}_v\,(1-\widehat{p}_v)}{T\,N}}
\]

is an unbiased estimator of the standard error, giving the
\((1-\alpha)\)\,--\,level normal confidence interval

\[
  \bigl[
    \widehat{p}_v - z_{1-\alpha/2}\,\widehat{\sigma}_v,\;
    \widehat{p}_v + z_{1-\alpha/2}\,\widehat{\sigma}_v
  \bigr],
  \qquad
  z_{1-\alpha/2}\in\{1.96,\,2.58,\dots\}.
\]

The tally routine therefore needs to maintain only the scalar
\(s_v\) while the simulation is running; the derived statistics can be updated
in-place whenever a user requests intermediate results or at a fixed cadence.

\subsection{Parallel accumulation algorithm}

The accumulation kernel is invoked on a three-dimensional
\texttt{nd\_range}, chosen such that
\[
  \begin{aligned}
    \text{global}_x &\;\ge\; V,\\
    \text{global}_y &\;\ge\; B,\\
    \text{global}_z &\;\ge\; P.
  \end{aligned}
\]
Work-item \((i_x,i_y,i_z)\) is responsible for \emph{exactly one} bit-pack:
\[
  \text{node  } v=i_x,\quad
  \text{batch } b=i_y,\quad
  \text{pack  } p=i_z.
\]

\vspace{4pt}
\noindent
\textbf{Local workflow of a work-item}
\begin{enumerate}
    \item Load the \(p^{\text{th}}\) bit-pack of batch \(b\) from
          \texttt{buffer}.
    \item Compute \(c=\mathrm{popcount}(\text{bitpack})\).
    \item Reduce the \(c\)’s belonging to the same work-\emph{group} in
          shared memory (tree reduction or \texttt{reduce\_over\_group}).
    \item One designated leader performs
          \(\texttt{atomic\_add}(\texttt{num\_one\_bits},\,\text{group\_sum})\).
\end{enumerate}

The reduction ensures only one atomic operation per group, greatly reducing
contention when \(P\) is large.

We present platform-neutral pseudocode that encapsulates the above logic while remaining agnostic to the underlying API.

\paragraph{Host-side control.}
After each Monte-Carlo iteration the host enqueues \textsc{TallyKernel} with a
fresh \texttt{iteration} counter.  When either (i)~a user requests
intermediate statistics or (ii)~a pre-set reporting interval is reached,
the host reads back \texttt{num\_one\_bits} and executes the purely
serial routine shown in Algorithm~\ref{alg:update_stats}.

\begin{algorithm}[H]
\caption{Post-processing of a single node’s tally}
\label{alg:update_stats}
\begin{algorithmic}[1]
  \Require
    \(s\) - total one-bits,
    \(T\), \(B\), \(P\), \(\omega\) - run parameters
  \Ensure
    \(\widehat{p}\), \(\widehat{\sigma}\), two symmetric CIs
  \State $N\gets B\cdot P\cdot\omega$
  \State $\widehat{p}\gets s / (T\,N)$
  \State $\widehat{\sigma}\gets
          \sqrt{\widehat{p}(1-\widehat{p})/(T\,N)}$
  \For{\textbf{each} $z\in\{1.96,\,2.58\}$}
      \State $\text{CI}\gets
        \bigl[\max(0,\widehat{p}-z\widehat{\sigma}),
              \min(1,\widehat{p}+z\widehat{\sigma})\bigr]$
  \EndFor
\end{algorithmic}
\end{algorithm}

The above normal approximation is valid provided \(T\,N\widehat{p}\)
and \(T\,N(1-\widehat{p})\) both exceed roughly 10; otherwise an exact
Clopper-Pearson interval can be substituted with no change to the running
sum logic.

\subsection{Correctness and complexity}

\textbf{Work-item cost.}
Each work-item performs one \(\mathrm{popcount}\) and
participates in an \(O(\log L)\) intra-group reduction
(\(L\!=\!\text{local\_range}\)), yielding an overall
\(O(\log L)\) instruction count.

\textbf{Global cost.}
The total number of work-items launched per iteration is
\(V\cdot B\cdot P\).  Because each bit-pack contains \(\omega\) Bernoulli
trials, the cost \emph{per trial} shrinks as \(\omega^{-1}\).

\textbf{Memory traffic.}
Every work-item reads exactly one machine word and no writes occur except
the single atomic addition per work-group.  Hence the algorithm is
memory-bandwidth bound only at extremely low arithmetic intensity
(\(P\approx 1\)).

\textbf{Linear scalability.}
All tally nodes are independent.  Increasing \(V\) therefore scales the total
runtime linearly until either (i)~the device saturates its occupancy or
(ii)~atomic contention becomes non-negligible; the group-level reduction
mitigates the latter.