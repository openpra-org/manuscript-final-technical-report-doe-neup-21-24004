\subsubsection{Method for Obtaining Minimal Cut Sets (MOCUS)}

Originally proposed by Fussell and Vesely in 1974\,\cite{Fussell1974MOCUS}, the \acrshort{mocus} algorithm remains one of the most widely deployed techniques for top-down generation of minimal cut sets in \acrshort{pra} tools. The procedure starts from the \textsc{top} event of a fault tree and repeatedly expands each logic gate until only basic events remain.

Let the following symbols be defined:
\begin{itemize}
  \item $w$ identifier of a logic gate
  \item $\varphi$ identifier of a basic event
  \item $\rho_{w,i}$ \emph{i}-th input to gate $w$
  \item $\lambda_{w}$ fan-in (number of inputs) of gate $w$
  \item $\Delta_{x,y}$ entry located at row $x$, column $y$ of the \acrfull{bics} matrix
  \item $x_{\max}$ ($y_{\max}$) current maximum row (column) index in~$\Delta$
\end{itemize}

\paragraph{Recursive expansion}
Starting with $\Delta_{1,1}=w_{\text{TOP}}$, every occurrence of a gate identifier is eliminated via the following transformation rules:
\begin{align}
  \Delta_{x,y}     &= \rho_{w,1}, \label{eq:mocus_seed} \\
  \Delta_{x,y_{\max}+1} &=
    \begin{cases}
      \rho_{w,\pi}, & \text{if } w \text{ is an AND gate},\\[4pt]
      \begin{cases}
        \Delta_{x,n}, & n=1,\dots,y_{\max},\; n\neq y,\\
        \rho_{w,\pi}, & n = y,
      \end{cases} & \text{if } w \text{ is an OR gate}.
    \end{cases}\label{eq:mocus_expand}
\end{align}
with $\pi=2,3,\dots,\lambda_w$.  
Equation~\eqref{eq:mocus_seed} seeds the matrix, while Equation~\eqref{eq:mocus_expand} generates new columns (AND) or rows (OR) until all $w$ symbols have been replaced by $\varphi$ symbols, thereby producing the complete set of BICS.

Two refinement steps convert \acrshort{bics} to \acrshort{mcs}:
\begin{enumerate}
  \item Remove duplicate basic events within each row.
  \item Discard any \acrshort{bics} that is a superset of another \acrshort{bics}.
\end{enumerate}

% \begin{figure}[ht]
\tikzset{
    grow'=down,
    level distance=42pt,
    % sibling distance=24pt,
    % sibling distance=8mm/#1,
    level/.style={sibling distance=3pt+44pt/(#1*#1*0.75)},
    edge from parent/.append style={
        draw,
        %thick,
        edge from parent path={
        (\tikzparentnode.west) |- ($(\tikzparentnode.west)!0.5!(\tikzchildnode.east)$) -| (\tikzchildnode.east)
        },
    },
    every node/.append style={
        anchor=center,
        rotate=90,
        draw=black,
        fill=white,
        thick,
        font=\footnotesize\bfseries,
        text centered,
        inner sep=0pt
    },
    var/.style={
        shape=circle,
        minimum height=16pt,
    },
    and/.style={
    and gate US,
    },
  not/.style={
    not gate US,
  },  
  or/.style={
    or gate US,
  },
  vot/.style={
    or gate US,
  }
}
  \centering
  \begin{tikzpicture}[
    circuit logic US,
    tiny circuit symbols,
    level 1/.style={sibling distance=40mm},
    level 2/.style={sibling distance=20mm},
    level 3/.style={sibling distance=20mm}
    ]

    \node[and]{\rotatebox{-90}{$G_1$}}
    child { node[or] {\rotatebox{-90}{$G_2$}}
      child { node[var] {\rotatebox{-90}{$X_1$}} }
      child { node[and] {\rotatebox{-90}{$G_3$}}
        child { node[var] {\rotatebox{-90}{$X_2$}} }
        child { node[var] {\rotatebox{-90}{$X_3$}} }
      }
    }
    child { node[or] {\rotatebox{-90}{$G_4$}}
      child { node[var] {\rotatebox{-90}{$X_2$}} }
      child { node[var] {\rotatebox{-90}{$X_4$}} }
    };
  \end{tikzpicture}
  \caption[Sample fault tree for MOCUS demonstration.]{Sample fault tree for MOCUS demonstration \cite{aras_dissertation_2024}.}
  \label{fig:sample_mocus_ft}
\end{figure}


% \begin{figure}[h!]
    \tikzset{
    grow'=down,
    level distance=42pt,
    % sibling distance=24pt,
    % sibling distance=8mm/#1,
    level/.style={sibling distance=3pt+44pt/(#1*#1*0.75)},
    edge from parent/.append style={
        draw,
        %thick,
        edge from parent path={
        (\tikzparentnode.west) |- ($(\tikzparentnode.west)!0.5!(\tikzchildnode.east)$) -| (\tikzchildnode.east)
        },
    },
    every node/.append style={
        anchor=center,
        rotate=90,
        draw=black,
        fill=white,
        thick,
        font=\footnotesize\bfseries,
        text centered,
        inner sep=0pt
    },
    var/.style={
        shape=circle,
        minimum height=16pt,
    },
    and/.style={
    and gate US,
    },
  not/.style={
    not gate US,
  },  
  or/.style={
    or gate US,
  },
  vot/.style={
    or gate US,
  }
}
    \centering
    \resizebox{\textwidth}{!}{%
\begin{tikzpicture}[
    circuit logic US,
    tiny circuit symbols,
    level 1/.style={sibling distance=5*18pt},
    level 2/.style={sibling distance=18pt},
    ]
  \node[or]{\rotatebox{-90}{Y}}
  % Subsets of size 3
  child { node[and] {} 
    child { node[var] {\rotatebox{-90}{$X_1$}} }
    child { node[var] {\rotatebox{-90}{$X_2$}} }
    child { node[var] {\rotatebox{-90}{$X_3$}} }
  }
  child { node[and] {} 
    child { node[var] {\rotatebox{-90}{$X_1$}} }
    child { node[var] {\rotatebox{-90}{$X_2$}} }
    child { node[var] {\rotatebox{-90}{$X_4$}} }
  }
  child { node[and] {} 
    child { node[var] {\rotatebox{-90}{$X_1$}} }
    child { node[var] {\rotatebox{-90}{$X_2$}} }
    child { node[var] {\rotatebox{-90}{$X_5$}} }
  }
  child { node[and] {} 
    child { node[var] {\rotatebox{-90}{$X_1$}} }
    child { node[var] {\rotatebox{-90}{$X_3$}} }
    child { node[var] {\rotatebox{-90}{$X_4$}} }
  }
  child { node[and] {} 
    child { node[var] {\rotatebox{-90}{$X_1$}} }
    child { node[var] {\rotatebox{-90}{$X_3$}} }
    child { node[var] {\rotatebox{-90}{$X_5$}} }
  }
  child { node[and] {} 
    child { node[var] {\rotatebox{-90}{$X_1$}} }
    child { node[var] {\rotatebox{-90}{$X_4$}} }
    child { node[var] {\rotatebox{-90}{$X_5$}} }
  }
  child { node[and] {} 
    child { node[var] {\rotatebox{-90}{$X_2$}} }
    child { node[var] {\rotatebox{-90}{$X_3$}} }
    child { node[var] {\rotatebox{-90}{$X_4$}} }
  }
  child { node[and] {} 
    child { node[var] {\rotatebox{-90}{$X_2$}} }
    child { node[var] {\rotatebox{-90}{$X_3$}} }
    child { node[var] {\rotatebox{-90}{$X_5$}} }
  }
  child { node[and] {} 
    child { node[var] {\rotatebox{-90}{$X_2$}} }
    child { node[var] {\rotatebox{-90}{$X_4$}} }
    child { node[var] {\rotatebox{-90}{$X_5$}} }
  }
  child { node[and] {} 
    child { node[var] {\rotatebox{-90}{$X_3$}} }
    child { node[var] {\rotatebox{-90}{$X_4$}} }
    child { node[var] {\rotatebox{-90}{$X_5$}} }
  }
  % Subsets of size 4
  child { node[and] {} 
    child { node[var] {\rotatebox{-90}{$X_1$}} }
    child { node[var] {\rotatebox{-90}{$X_2$}} }
    child { node[var] {\rotatebox{-90}{$X_3$}} }
    child { node[var] {\rotatebox{-90}{$X_4$}} }
  }
  child { node[and] {} 
    child { node[var] {\rotatebox{-90}{$X_1$}} }
    child { node[var] {\rotatebox{-90}{$X_2$}} }
    child { node[var] {\rotatebox{-90}{$X_3$}} }
    child { node[var] {\rotatebox{-90}{$X_5$}} }
  }
  child { node[and] {} 
    child { node[var] {\rotatebox{-90}{$X_1$}} }
    child { node[var] {\rotatebox{-90}{$X_2$}} }
    child { node[var] {\rotatebox{-90}{$X_4$}} }
    child { node[var] {\rotatebox{-90}{$X_5$}} }
  }
  child { node[and] {} 
    child { node[var] {\rotatebox{-90}{$X_1$}} }
    child { node[var] {\rotatebox{-90}{$X_3$}} }
    child { node[var] {\rotatebox{-90}{$X_4$}} }
    child { node[var] {\rotatebox{-90}{$X_5$}} }
  }
  child { node[and] {} 
    child { node[var] {\rotatebox{-90}{$X_2$}} }
    child { node[var] {\rotatebox{-90}{$X_3$}} }
    child { node[var] {\rotatebox{-90}{$X_4$}} }
    child { node[var] {\rotatebox{-90}{$X_5$}} }
  }
  % Subset of size 5
  child { node[and] {} 
    child { node[var] {\rotatebox{-90}{$X_1$}} }
    child { node[var] {\rotatebox{-90}{$X_2$}} }
    child { node[var] {\rotatebox{-90}{$X_3$}} }
    child { node[var] {\rotatebox{-90}{$X_4$}} }
    child { node[var] {\rotatebox{-90}{$X_5$}} }
  };
\end{tikzpicture}
    }
    \caption{3-of-5 voting logic, expanded as (AND-OR) \acrfull{dnf}}
    \label{fig:example-3of5-voter-tree}
\end{figure}


Table \ref{tab:sample_mocus_ft_data} summarizes the sample fault tree by matching the formulas provided for the algorithm.

\begin{table}[htbp]
  \centering
  \caption{Structural data for the sample fault tree in Figure~\ref{fig:sample_mocus_ft}.}
  \label{tab:sample_mocus_ft_data}
  \begin{tabular}{@{}lccc@{}}
    \toprule
    $w$ & Gate type & $\lambda_{w}$ & $\rho_{w,i}$ \\ \midrule
    $G_1$ & AND & 2 & $G_2$\;\;$G_4$ \\
    $G_2$ & OR  & 2 & $X_1$\;\;$G_3$ \\
    $G_4$ & OR  & 2 & $X_2$\;\;$X_4$ \\
    $G_3$ & AND & 2 & $X_2$\;\;$X_3$ \\ \bottomrule
  \end{tabular}
\end{table}

Figure~\ref{fig:mocus_flow} schematically depicts the evolution of the $\Delta_{x,y}$ matrix for the fault tree in Figure \ref{fig:sample_mocus_ft}. 

After gate expansion, elimination of duplicates (e.g.\ event $X_2$) and removal of supersets (e.g.\ $\{X_2,X_4,X_3\}$) yield the \acrshort{mcs} family:
\[
  \bigl\{\,\{X_1,X_2\},\; \{X_2,X_3\},\; \{X_1,X_4\}\,\bigr\}.
\]

\begin{figure}[h!]
  \includesvg[width=\textwidth]{2_foundations/quantification/qualitative/figures/mocus_flow.svg}
  \caption{Schematic representation of MOCUS algorithm application to the sample fault tree.}
  \label{fig:mocus_flow}
\end{figure}

Although algorithmically elegant, the recursive nature of \acrshort{mocus} can incur considerable computational overhead for large, deeply nested fault trees.  Nonetheless, virtually every \acrshort{pra} tool integrates some variant of \acrshort{mocus} for minimal cut-set calculation.
