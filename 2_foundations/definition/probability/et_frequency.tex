\section{Quantifying Event Sequences}

Because risk analysis requires knowing how likely each branch in the tree is, event trees rely heavily on \emph{conditional probabilities}. Let
\[
    p(I)
    \;\equiv\;
    \Pr(I)
\]
be the probability (or frequency) of the initiating event. For each functional event \(F_k\), define
\[
    p\bigl(F_k^{\text{succ}}\mid I,\, F_1,\,\dots,\,F_{k-1}\bigr)
    \quad\text{and}\quad
    p\bigl(F_k^{\text{fail}}\mid I,\, F_1,\,\dots,\,F_{k-1}\bigr),
\]
which describe the likelihood of success or failure given all prior outcomes.

An \emph{end-state} \(X_j\) arises from a particular chain of successes/failures:
\[
    \bigl(I,\,F_1^{\alpha_1},\,F_2^{\alpha_2},\,\ldots,\,F_n^{\alpha_n}\bigr)
    \;\longrightarrow\; 
    X_j,
\]
where each \(\alpha_k \in \{\text{succ},\,\text{fail}\}\). The probability of reaching \(X_j\) is the product of:
\begin{enumerate}
    \item The initiating event probability \(p(I)\).
    \item The conditional probabilities of each functional event's success or failure.
\end{enumerate}
Formally, if \(\omega_j\) denotes the entire branch leading to end-state \(X_j\), then
\begin{align}
\label{eq:event_tree_branch_probability}
    p(\omega_j)
    \;=\;
    p(I)
    \times
    \prod_{k=1}^{n}\,
    p\!\bigl(F_k^{\alpha_k}\mid 
             I,\,
             F_1^{\alpha_1},\ldots,
             F_{k-1}^{\alpha_{k-1}}\bigr).
\end{align}
The union of all such branches spans the full sample space of scenario outcomes generated by \(I\) and the subordinate functional events. Next, it is shown that every branch of an event tree can be represented by a product (logical AND) of the relevant Boolean variables for the initiating event and each functional event’s success/failure. Collecting all branches via logical OR yields a disjunction of these products, precisely matching the standard structure of a Boolean expression in \acrfull{dnf}. This idea is explored further in Section~\ref{sec:event_trees_as_2_lvl_circuits}.