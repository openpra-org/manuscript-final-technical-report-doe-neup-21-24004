% \subsection{Combining Shared Subtrees and Large Systems}

% In standard reliability analysis, certain fault tree computations—such as identifying \acrfull{mcs} or calculating the probability of the top event—can be done by enumerating combinations of basic events that ensure system failure. However, if large shared subtrees exist or many gates gather multiple inputs, the expansion can become combinatorially challenging. Nevertheless, the \acrshort{dag} structure remains powerful for:

% \begin{itemize}
% \item \textbf{Clarity and Modularity:} Decomposing the system or subsystem failures into smaller, well-defined parts.  
% \item \textbf{Reuse of Subtrees:} Representing components or subsystems that are relevant to multiple parts of the fault logic without arbitrarily duplicating them.  
% \item \textbf{Analytical Efficiency:} Exploiting independence or bounding certain events, thereby limiting the combinatorial explosion.  
% \end{itemize}
