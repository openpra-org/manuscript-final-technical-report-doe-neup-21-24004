\subsection{Decision Diagrams}
\label{sec:decision_diagrams}

Decision diagrams provide a powerful, directed-graph-based representation of logical or Boolean functions. Their roots can be traced back to branching program ideas explored by Lee and Akers in the 1950s–1970s, but major refinement and widespread adoption occurred after Bryant's seminal work on \emph{Ordered Binary Decision Diagrams (OBDDs)} in 1986. In reliability analysis, formal verification, and combinational circuit design, decision diagrams frequently offer more computationally tractable methods than naïve enumeration of all input patterns.

This section introduces the basic concepts of \emph{Binary Decision Diagrams (BDDs)} and \emph{Zero-Suppressed Decision Diagrams (ZDDs)}, along with the special class of \emph{Ordered} and \emph{Reduced} BDDs that guarantee a canonical (unique) form under fixed conditions. We emphasize:
\begin{itemize}
\item The structure and interpretation of BDDs as directed acyclic graphs (DAGs).
\item The notion of an \emph{ordered} BDD, imposing a strict arrangement on variable testing.
\item Techniques for \emph{reducing} BDDs into smaller yet equivalent graphs by merging or removing redundant parts.
\item The main principles of Zero-Suppressed Decision Diagrams, designed for efficiently encoding sparse sets or combinatorial families.
\end{itemize}

Earlier, we saw that \emph{event trees} and \emph{fault trees} can be merged into a single directed acyclic graph to represent complex system dependencies. BDDs and ZDDs, by contrast, focus more narrowly on Boolean functions, providing specialized node-splitting and merging operations to systematically capture logical behavior. Despite differing motivations, both families of DAG-based representations benefit from the avoidance of cycles and the ability to encode large models in a structured form.


\subsubsection{Binary Decision Diagrams (BDD)}
\label{sec:bdd}

Let $f: \{0,1\}^n \to \{0,1\}$ be an $n$-variable Boolean function. A \acrfull{bdd} is a directed acyclic graph whose internal nodes represent decisions on a single Boolean variable, and whose terminal (sink) nodes represent constant outputs ($0$ or $1$).

\begin{definition}[\acrlong{bdd}]
\label{def:bdd}
A \acrlong{bdd} for $f$ is a tuple $B =\bigl(N,n_0,V,E,T\bigr)$ with the following components:
\begin{enumerate}
\item $N$ is a finite set of nodes, partitioned into \emph{internal} nodes and \emph{terminal} nodes.
\item $n_0 \in N$ is the \emph{root node}, where evaluations begin.
\item $V={x_1,x_2,\dots,x_n}$ is the set of Boolean variables associated with the internal nodes.
\item $E \subseteq N \times {0,1} \times N$ is the edge set. Each internal node $u$ has two labeled edges, $(u,0,v_0)$ and $(u,1,v_1)$, indicating the next node in the diagram if $x_i=0$ or $x_i=1$ at node $u$.
\item $T$ is a mapping that assigns the value $0$ or $1$ to each terminal node of $B$.
\end{enumerate}
For any input $a = (a_1,a_2,\dots,a_n)\in\{0,1\}^n$ one identifies a unique path from $n_0$ to a terminal node by at each internal node following the edge labeled by the tested variable’s value in $a$. The value of $f(a)$ is given by the terminal node reached, as encoded by $T$.
\end{definition}

\noindent\textbf{Interpretation.} Each internal node corresponds to a variable test: if the variable is $0$ (i.e.\ $\text{false}$), follow the \emph{0-edge}, and if it is 1 $(\text{true})$, follow the \emph{1-edge}. Eventually, one reaches a sink node labeled $\text{false}=0$ or $\text{true}=1$.

\paragraph{Example.} For the three-variable function
$f(a,b,c)=a \land \bigl(b \lor c\bigr)$,
Figure~\ref{fig:example_bdd} shows a small \acrshort{bdd}. Each circular node tests one variable $a$, $b$, or $c$; the dashed and solid edges denote the 0- and 1-branches, respectively. Terminal nodes (squares) contain a 0 or 1 label.

\begin{figure}[htbp]
\centering
\begin{tikzpicture}[->,>=stealth',shorten >=1pt,auto,node distance=2.1cm,thick,
main node/.style={circle,draw,font=\sffamily\small\bfseries},
terminal node/.style={rectangle,draw,font=\sffamily\small\bfseries}]

\node[main node] (A) {a};
\node[main node] (B) [below left=1.6cm of A, xshift=0.6cm] {b};
\node[main node] (C) [below right=1.6cm of A, xshift=-0.6cm] {c};
\node[terminal node] (T1) [below=1.8cm of B] {1};
\node[terminal node] (T0) [below=1.8cm of C] {0};

\path[every node/.style={font=\sffamily\scriptsize}]
(A) edge[dashed] node[left]  {0} (T0)
edge[solid]  node[right] {1} (B)
(B) edge[dashed] node[left]  {0} (C)
edge[solid]  node[right] {1} (T1)
(C) edge[dashed] node[left]  {0} (T0)
edge[solid]  node[right] {1} (T1);

\end{tikzpicture}
\caption{A \acrfull{bdd} for $f(a,b,c)=a \land \bigl(b \lor c\bigr)$.}
\label{fig:example_bdd}
\end{figure}

In practice, \acrshort{bdd}s may experience large variations in size depending on how variables are tested as one traverses the graph. The \textit{ordered} and \textit{reduced} variants of \acrshort{bdd}s are especially important, as they yield canonical forms for fixed variable orderings.

\begin{definition}[\acrfull{obdd}]
\label{def:obdd}
An \acrfull{obdd} imposes a strict ordering $\pi$ on the variables ${x_1,\dots,x_n}$. For every path from the root to a terminal node, if the path encounters variables $x_i$ and $x_j$, then $x_i$ is tested before $x_j$ whenever $i<j$ with respect to $\pi$. Equivalently, no path may test a higher-indexed variable and later test a lower-indexed one.
\end{definition}

\acrshort{obdd}s are also known as \emph{read-once branching programs with an ordering restriction.} Bryant showed that under a particular variable order, the representation is often more compact than arbitrary \acrshort{bdd}s and that many operations (e.g., equivalence checking, conjunction, disjunction) can be carried out efficiently.

\begin{definition}[[\acrfull{robdd}]
\label{def:reducedobdd}
An \acrshort{obdd} is said to be \emph{reduced} if it contains no isomorphic subgraphs and no node whose 0- and 1-branches lead to the exact same child. Equivalently, one applies two \textit{reduction rules}:
\begin{enumerate}
\item \textbf{Elimination Rule:} If, for a given node $v$, the 0-edge and 1-edge both point to the same successor, remove $v$ and connect its incoming edges directly to that successor.
\item \textbf{Merging Rule:} If two distinct nodes $u$ and $v$ test the same variable and have identical 0- and 1-successors, merge them into a single node.
\end{enumerate}
A \emph{\acrfull{robdd}} respects a global variable order $\pi$ and has been minimized via these rules.
\end{definition}

\textbf{Canonical Representation.} One of the principal advantages of $\mathrm{ROBDD}$s is that, for a fixed variable ordering, every Boolean function has a unique representation. Consequently, checking whether two functions are identical reduces to testing whether their $\mathrm{ROBDD}$s coincide as node- and edge-labeled graphs.

\begin{theorem}[Canonical Form of \acrshort{robdd}s]
\label{thm:canonical}
Let $\pi$ be a fixed ordering on the variables ${x_1,\dots,x_n}$. Then for any Boolean function $f$ of $n$ variables, its reduced \acrshort{obdd} with respect to $\pi$ is unique.
\end{theorem}

A direct consequence is that striving for reduced ordered forms both shrinks redundant structure and supports robust equivalence checks.

\subsubsection{Zero-Suppressed Decision Diagrams (ZDD)}
\label{sec:zdd}

For certain applications, notably combinatorial itemset enumeration and other \emph{sparse} set representations, \acrfull{zdd} can be more compact than standard \acrshort{bdd}s. Although \acrshort{zdd}s adhere to similar principles of node-based variable testing, they selectively \emph{omit} many zero-branches that do not yield new information.

\paragraph{Key Distinctions.}
While \acrshort{zdd}s also enforce an ordering and can be reduced via isomorphism checks, the core difference lies in the zero-suppression mechanism:
\begin{itemize}
\item If following a 0-edge provides no meaningful distinction in the final outcome, that 0-edge and its corresponding node are pruned.
\item The 1-branches are retained but merged where possible, much as in \acrshort{robdd}s.
\end{itemize}
By removing portions of the diagram where "nothing interesting" (i.e.\ no new sets or subsets) occurs, the diagram remains compact.

\paragraph{Illustrative Example}
Revisiting $f(a,b,c)=a \land \bigl(b \lor c\bigr)$ from above, Figure~\ref{fig:zdd} sketches a plausible \acrshort{zdd}. Note here:
\begin{itemize}
\item Node (a) splits into a 0-edge that immediately goes to a node (or directly to a \texttt{0}-terminal) that is pruned if it carries no unique set representation.
\item The 1-edge leads to further variable tests ($b$ or $c$), but many 0-branches are again suppressed if they do not alter the final outcome distinct from an already-represented path.
\end{itemize}

\begin{figure}[htbp]
\centering
\begin{tikzpicture}[->,>=stealth',shorten >=1pt,auto,node distance=2cm,thick,
main node/.style={circle,draw,font=\sffamily\small\bfseries},
terminal node/.style={rectangle,draw,font=\sffamily\small\bfseries}]

\node[main node] (A) {$a$};
\node[main node] (B) [below right of=A] {$b$};
\node[main node] (C) [below left of=B] {$c$};

\node[terminal node] (Z1) [below of=C] {1};
\node[terminal node] (Z0) [right of=Z1,node distance=2.8cm] {0};

\path[every node/.style={font=\sffamily\scriptsize}]
(A) edge[dashed] node[left]  {0} (Z0)
edge[solid]  node[right] {1} (B)
(B) edge[dashed] node[left]  {0} (C)
edge[solid]  node[right] {1} (Z1)
(C) edge[dashed] node[left]  {0} (Z0)
edge[solid]  node[right] {1} (Z1);

\end{tikzpicture}
\caption[A \acrfull{zdd} for $f(a,b,c)=a \land \bigl(b \lor c\bigr)$.]{A \acrfull{zdd} for $f(a,b,c)=a \land \bigl(b \lor c\bigr)$. Many zero-branches are pruned.}
\label{fig:zdd}
\end{figure}

In general, \acrshort{zdd}s apply much the same merging rules as \acrshort{robdd}s and can yield similarly unique structures for a given variable order. They tend to excel in representing large but sparsely populated families of subsets (e.g., all minimal cut sets in a reliability system) because superfluous 0-edges are systematically suppressed.
\subsubsection{Probabilistic Sentential Decision Diagrams (PSDD)}
\label{sec:psdd}

A \acrfull{psdd} is a tractable representation for a probability distribution over a set of propositional variables subject to logical constraints.  In essence, a \acrshort{psdd} is a parameterized \acrfull{sdd} in which each node is assigned a well-defined local distribution.  By construction, the \acrshort{psdd}’s global distribution respects a given \emph{base theory} (i.e., a propositional formula representing constraints), assigns zero probability to every assignment violating that theory, and factors the remaining assignments according to a hierarchical decomposition.

% this is incorrect, vtree defines \acrshort{sdd}, not in parallel
% To define a \acrshort{psdd} precisely, one must first choose:
% \begin{enumerate}
% \item A \emph{vtree}, which is a full binary tree whose leaves are in bijection with the set of variables.  The internal nodes of the vtree determine how variables are split and combined at different levels of the diagram.
% \item A \emph{normalized \acrshort{sdd}} encoding the same propositional constraints as the original theory.  An \acrshort{sdd} is built from \emph{decision nodes} and \emph{terminal} (leaf) nodes:
% \begin{itemize}
% \item A \emph{decision node} has elements of the form $(p_i, s_i)$, each comprising two $sub-$SDDs: a \emph{prime}$p_i$ and a \emph{sub}$s_i$.
% \item A \emph{terminal node} is simply a literal (e.g., $X$, $\neg X$) or a constant ($\top$ or $\bot$).
% \end{itemize}
% Normalization requires that each decision node’s primes refer to one vtree subtree and subs refer to a disjoint subtree, reflecting how variables are partitioned at each vtree node.
% \end{enumerate}

\begin{definition}[Normalized \acrshort{sdd}]
Given a \emph{vtree}, $v$, over variables ${X_1, ... , X_n}$, an normalized \acrshort{sdd} over $T$ is defined recursively as follows:
\begin{enumerate}
    \item A \emph{terminal} node is a literal $X_i$ or $\neg X_i$
\item At an \emph{internal vtree node} with left subtree $V_l$ and right subtree $V_r$, a normalized \acrshort{sdd} is a finite disjunction: $$\bigvee_{i=1}^k \left( P_i \land S_i \right)$$ where:
\begin{enumerate}
    \item For every $i$,  $P_i$ is a normalized \acrshort{sdd} over the variables in $V_l$, and $S_i$ is a normalized \acrshort{sdd} over the variables in $V_l$.
    \item The set $\{P_1, \ldots, P_k\}$ forms a partition of the space over $V_l$; that is, $\forall i \neq j: P_i \land P_j = \bot$ (mutually exclusive) and $\bigvee_{i=1}^k P_i = \top$(exhaustive).
    \item Each $S_i$ is distinct (no two are equivalent).
    \item Each prime-sub pair $(P_i, S_i)$ is compressed: distinct primes never associate to equivalent subs and vice versa (no redundancy).
\end{enumerate}

\end{enumerate}
\end{definition}

Once the normalized \acrshort{sdd} is fixed, one may introduce continuous parameters to obtain a \acrshort{psdd}. These parameters, in effect, turn each decision node into a \emph{local mixture} of its prime components, while terminal nodes over variables become Bernoulli distributions.

\begin{definition}[\acrshort{psdd} Syntax]
\label{def:psdd-syntax}
Let $n$ be an \acrshort{sdd} node normalized for a vtree node $v$.
\begin{itemize}
\item If $n$ is a terminal node:
\begin{enumerate}
\item If it encodes a literal (e.g.\ $X$, $\neg X$) or the constant~$\bot$, then its probability is fixed implicitly (e.g.\ $\neg X$ yields $\Pr(\neg X)=1,$ $\Pr(X)=0$ for that node alone).
\item If it is the constant~$\top$ and $v$ corresponds to some variable $X$, then we assign a parameter $\theta\in(0,1)$ indicating $\Pr(X)$ at this node.
\end{enumerate}
\item If $n$ is a decision node with $k$ elements $[(p_1,s_1), \ldots, (p_k,s_k)]$, we assign nonnegative parameters $\theta_1,\ldots,\theta_k$ such that $\sum_{i=1}^k \theta_i=1$.  Furthermore, if $s_i = \bot$, then $\theta_i$ must be zero (no probability is allotted to a sub whose base is unsatisfiable).
\end{itemize}
The resulting parameterized structure is called a \emph{\acrshort{psdd}}.
\end{definition}

Each node $n$ in a \acrshort{psdd} induces a local distribution $\Pr_n(\cdot)$ on the variables of the vtree node $n$ is normalized for.  At a decision node, the probability of a complete assignment is given by multiplying:
\begin{enumerate}
\item The probability that we "choose" a particular prime~$p_i$, labeled by $\theta_i$.
\item The probability contributed by prime node~$p_i$ on its variables.
\item The probability contributed by sub node~$s_i$ on its (disjoint) variables.
\end{enumerate}
Summing across all primes yields $\Pr_n$ at that node.  By construction, the \emph{root} node’s distribution~$\Pr_r$ then covers all variables and zeroes out any assignments that do not satisfy the base theory.

\begin{theorem}[Base Property \cite{darwiche_knowledge_2002}]
\label{thm:psdd-base}
If a \acrshort{psdd} node $n$ is normalized for vtree node~$v$, then $\Pr_n(x)=0$ whenever $x$ does not satisfy the \acrshort{sdd} sub-formula~$[n]$.  At the root node $r$, $\Pr_r(x)>0$ only if $x$ satisfies the entire theory.
\end{theorem}

\paragraph{Parameter Semantics.} A key property of \acrshort{psdd}s is that each parameter $\theta_i$ can be interpreted \emph{locally} as a conditional probability given the \emph{context} of the decision node.  Formally, if node~$n$ has context~$\gamma_n$ (i.e., the partial assignment implied by traversing the \acrshort{sdd} from the root to~$n$), then:
\[
\theta_i =\Pr\bigl([p_i] \bigm|[\gamma_n]\bigr)
\]
where $[,p_i]$ is the logical content of prime $p_i$.  This ensures that local parameters align with global $\Pr(\cdot)$ in a transparent, compositional way.

\paragraph{Context-Specific Independence.} Due to the vtree-based factorization, \acrshort{psdd}s capture rich \emph{context-specific independences} \cite{boutilier_context-specific_2013}.  At high level, once we know the node’s context (which is a partial assignment or formula), certain subsets of variables become conditionally independent of the rest.  These independence statements can be read directly from the \acrshort{psdd} structure, generalizing common conditional-independence ideas in Bayesian or Markov networks.

\paragraph{Inference and Tractability.} A central advantage of \acrshort{psdd}s is that computing $(\Pr_r(e)$ for any evidence~$e$ can be done in time linear in the size of the \acrshort{psdd}.  This incremental algorithm proceeds bottom-up through each node, locally aggregating evidence contributions and summing accordingly.  Moreover, once node-level evidence statistics are available, one can also efficiently compute single-variable or pairwise marginals using a second top-down pass.

\paragraph{Parameter Learning under Complete Data.} Another appealing property is that \emph{maximum-likelihood} parameters can be determined in closed form when every training example is a complete assignment of all variables.  Specifically, if a \acrshort{psdd} node $n$ with context~$\gamma_n$ has elements $(p_i, s_i)$, one sets
\[
\theta_i = \frac{\text{number of data points satisfying both } \gamma_n \text{ and } p_i}
{\text{number of data points satisfying } \gamma_n}.
\]
Parallel rules apply for terminal nodes representing $\top$.  Because sub-contexts $\gamma_n \wedge p_i$ are pairwise disjoint, it suffices to tabulate data counts for each feasible sub-context.  The outcome is a simple frequency-based update analogous to parameter estimation in Bayesian networks, yet here it respects the underlying \acrshort{sdd} constraints exactly.

For any propositional distribution (and chosen vtree), there exists a corresponding \acrshort{psdd} whose root distribution matches it exactly.  Furthermore, if the \acrshort{psdd} is kept \emph{compressed} (meaning no redundant substructures), this representation is \emph{unique} up to isomorphic details \cite{darwiche_knowledge_2002}.  Thus, \acrshort{psdd}s can serve as canonical forms for distributions under logical constraints.

In contrast to classical graphical models, \acrshort{psdd}s operate at the confluence of tractable \emph{Boolean} structure (via \acrshort{sdd}s) and probability theory.  They explicitly encode zero-probability assignments (via the \acrshort{sdd} base) while ensuring all positive assignments factor through the decision nodes.  Their parameter semantics aligns each local weight with a well-defined global conditional probability.  In addition, closed-form parameter learning is possible in the complete-data setting.  Hence, \acrshort{psdd}s provide a principled, canonical choice for modeling distributions when the domain is governed by complex logical constraints.
