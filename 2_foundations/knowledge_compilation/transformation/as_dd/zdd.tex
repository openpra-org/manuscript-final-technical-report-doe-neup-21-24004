\subsubsection{Zero-Suppressed Decision Diagrams (ZDD)}
\label{sec:zdd}

For certain applications, notably combinatorial itemset enumeration and other \emph{sparse} set representations, \acrfull{zdd} can be more compact than standard \acrshort{bdd}s. Although \acrshort{zdd}s adhere to similar principles of node-based variable testing, they selectively \emph{omit} many zero-branches that do not yield new information.

\paragraph{Key Distinctions.}
While \acrshort{zdd}s also enforce an ordering and can be reduced via isomorphism checks, the core difference lies in the zero-suppression mechanism:
\begin{itemize}
\item If following a 0-edge provides no meaningful distinction in the final outcome, that 0-edge and its corresponding node are pruned.
\item The 1-branches are retained but merged where possible, much as in \acrshort{robdd}s.
\end{itemize}
By removing portions of the diagram where "nothing interesting" (i.e.\ no new sets or subsets) occurs, the diagram remains compact.

\paragraph{Illustrative Example}
Revisiting $f(a,b,c)=a \land \bigl(b \lor c\bigr)$ from above, Figure~\ref{fig:zdd} sketches a plausible \acrshort{zdd}. Note here:
\begin{itemize}
\item Node (a) splits into a 0-edge that immediately goes to a node (or directly to a \texttt{0}-terminal) that is pruned if it carries no unique set representation.
\item The 1-edge leads to further variable tests ($b$ or $c$), but many 0-branches are again suppressed if they do not alter the final outcome distinct from an already-represented path.
\end{itemize}

\begin{figure}[htbp]
\centering
\begin{tikzpicture}[->,>=stealth',shorten >=1pt,auto,node distance=2cm,thick,
main node/.style={circle,draw,font=\sffamily\small\bfseries},
terminal node/.style={rectangle,draw,font=\sffamily\small\bfseries}]

\node[main node] (A) {$a$};
\node[main node] (B) [below right of=A] {$b$};
\node[main node] (C) [below left of=B] {$c$};

\node[terminal node] (Z1) [below of=C] {1};
\node[terminal node] (Z0) [right of=Z1,node distance=2.8cm] {0};

\path[every node/.style={font=\sffamily\scriptsize}]
(A) edge[dashed] node[left]  {0} (Z0)
edge[solid]  node[right] {1} (B)
(B) edge[dashed] node[left]  {0} (C)
edge[solid]  node[right] {1} (Z1)
(C) edge[dashed] node[left]  {0} (Z0)
edge[solid]  node[right] {1} (Z1);

\end{tikzpicture}
\caption[A \acrfull{zdd} for $f(a,b,c)=a \land \bigl(b \lor c\bigr)$.]{A \acrfull{zdd} for $f(a,b,c)=a \land \bigl(b \lor c\bigr)$. Many zero-branches are pruned.}
\label{fig:zdd}
\end{figure}

In general, \acrshort{zdd}s apply much the same merging rules as \acrshort{robdd}s and can yield similarly unique structures for a given variable order. They tend to excel in representing large but sparsely populated families of subsets (e.g., all minimal cut sets in a reliability system) because superfluous 0-edges are systematically suppressed.