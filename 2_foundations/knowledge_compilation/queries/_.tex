\section{Queries}
The primary choice that dictates the selection of target languages in \acrshort{pra} is availability of tractable/executable queries for a given language. The two most important query types that are of interest to \acrshort{pra} are Model Counting (CT) and Model Enumeration (ME).

\subsection{Model Counting}
Model Counting, a.k.a $\#$SAT-problem, is a classic problem in Computability Theory and VLSI (Very Large Scale Integration) analysis. It strive to answer the following question about a given boolean expression (theory): ``How many valid true/false variable assignments (models) satisfy the boolean expression?'' This problem is central to \acrshort{pra}, as it the compuation algorithm behind Weighted Model Counting approach used for computing probabilities of top events in Fault Trees and End States in Even Sequence Diagrams. 

Importance of this query has largely determined the relative scarcity of underlying implementations of \acrshort{pra} software. Tractability of this query in OBDD family of target languages is primarily reason for popularity of this data structure among \acrshort{pra} software vendors. However, despite the its perceived NP-complexity, CNF-based $\#$SAT-solvers can also be used, due to their efficient implementations and long-running research. Finally, with new research showing still flowing, SDDs, identify themselves as a promising alternative due to their improved succinctness.

\subsection{Model Enumeration}
Model enumeration is arguably even more important query for \acrshort{pra}, as it responsible for implementations exhaustive Minimal Cutset search. This is one of the most intractable algorithms allowing for only polynomial-time delay between successive solution, as the primary determining complexity factor is exponential ``blow-up'' of the number of cutsets in the first place. Few implementations, offer truly efficient and tractable solutions, with CNFs and ROBDDs yet again taking the spotlight.


% % \subsection{\color{blue}{Expectation}}
% \subsubsection{\color{blue}{Probability Computation as a Function Problem}}
% \subsubsection{\color{blue}{The \#P Complexity Class}}
% \subsubsection{\color{blue}{Relationship to Weighted Model Counting}}
% \subsection{Structural Enumeration}