\section{\color{blue}{Document Layout}}
This report is organized to provide a comprehensive and logical progression from foundational concepts to the technical realization and evaluation of the developed PRA platform. The document begins with an introduction that outlines the motivation, scope, objectives, and intended outcomes of the project, as well as the specific aims of this report.

The first major part, \textit{Foundations}, establishes the theoretical and methodological basis for probabilistic risk assessment. It covers the triplet definition of risk, formalizes event and fault tree structures, and introduces the unified probabilistic directed acyclic graph (PDAG) model. This section also discusses equivalent model representations, qualitative analysis techniques, and methods for quantifying probabilities and frequencies, including both exact and approximate approaches.

The second part, \textit{Identifying Gaps}, reviews the current state of PRA software, highlighting persistent limitations in scalability, model development, and risk communication. It details the development of benchmark models, including both generic and synthetic examples, and describes the process of translating and quantifying these models using existing tools. This part also presents benchmarking results and discusses the application of the platform to multi-hazard PRA models.

The third part, \textit{Proposed High-Throughput Compute Solution}, describes the architectural design and implementation of the new platform. It covers the overall system architecture, the design of the distributed queuing system, and the development of a data-parallel Monte Carlo probability estimator. This section also addresses performance evaluation, scalability, and integration with the web-based OpenPRA platform.

The final part, \textit{Future Work}, outlines ongoing and future research directions. It identifies current limitations, such as the need for enhanced collaboration, version control, solver integration, and advanced uncertainty quantification. The report concludes with a discussion of the steps required to advance the platform toward full industry readiness and broader community adoption.

Throughout the document, figures, tables, and equations are used to clarify technical concepts and present results. References to relevant literature and standards are provided to support the discussion and facilitate further exploration by the reader.
