\chapter{Introduction}

\section{\color{blue}{Project Motivation}}
The development and use of PRAs in the nuclear industry have revolutionized our approach to design, reliability, and safety. Since the release of WASH-1400 in 1975 and NUREG-1150 in 1990, the PRA applications have diversified and become more computationally demanding, nevertheless the tools that we use to perform such assessments have not kept up with the technological advancements in high-performance computing and computer science applications. Now, although there is a need for supporting real-time decisions at the nuclear plants using PRAs, this is not practical using the computationally strained legacy PRA tools currently available since they were developed and are maintained using technologies that are nowadays deprecated. The legacy PRA tools, although still perform well on internal events PRA models that have reasonable sizes, suffer greatly, both in terms of speed and memory requirements, when challenged by more sophisticated models such as single hazard PRAs, and especially multi-hazard PRAs. Therefore, a major redesign of the PRA tools is necessary starting from the computational engine capabilities, backend services to handle large PRA models, and a user-friendly PRA frontend that can automatically generate results and documentation necessary to inform non-PRA experts.

\section{\color{blue}{Project Scope}}
The scope of this project is to design, implement, and evaluate an open source, parallel, and distributed web-based PRA platform that addresses the limitations of legacy PRA tools. The platform is intended to support a wide range of PRA applications, from traditional internal events models to advanced multi-hazard and multi-unit scenarios. The project encompasses the development of a modular architecture that integrates a web-based client for model editing and visualization, RESTful backend APIs for data management and orchestration, and a distributed job queue system for high-throughput risk quantification. The platform is designed to be extensible, supporting integration with multiple quantification engines and enabling interoperability with industry-standard tools through the adoption of the OpenPRA schema and compatibility with the OpenPSA Model Exchange Format (MEF). The project also includes the development of automated benchmarking and model translation workflows, as well as the implementation of features to facilitate collaborative model development, transparent risk communication, and reproducible analysis. The ultimate goal is to provide a robust, scalable, and user-friendly solution that meets the evolving needs of PRA practitioners and stakeholders in the nuclear industry.

\section{\color{blue}{Project Objectives}}
The main objective of this work is to develop, demonstrate, and evaluate a probabilistic risk assessment (PRA) software platform needed to address the major challenges of the current legacy PRA tools, such as better quantification speed, integration of multi-hazard models into traditional PRAs, and model modification simplification and documentation automation. To achieve the main objective, we will first perform benchmarking and profiling of current PRA tools, such as SCRAM [1] and US NRC's SAPHIRE [2, Vol. 1] , to investigate the current bottlenecks in the quantification speed and memory requirements. Secondly, we will design, implement, and benchmark a PRA software platform based on a web-based stack using the latest technologies available to overcome the mentioned challenges. Finally, we will evaluate the performance gains of this framework by modeling and quantifying large PRA models that would have been too expensive to run using the legacy PRA tools.

\section{\color{blue}{Objectives of this Report}}
This report documents the design, implementation, and evaluation of the open source, parallel, and distributed web-based PRA platform developed under this project. The objectives of this report are to:
\begin{itemize}
    \item Present the motivation, scope, and objectives of the project in the context of current challenges in PRA technology.
    \item Describe the architectural design and technical implementation of the platform, including its modular components and integration strategies.
    \item Summarize the benchmarking and evaluation of the platform against existing PRA tools, highlighting improvements in computational efficiency, scalability, and model interoperability.
    \item Identify current limitations and outline a roadmap for future research and development to address remaining gaps and advance the platform toward industry readiness.
    \item Provide guidance and best practices for PRA practitioners and stakeholders interested in adopting or contributing to the platform.
\end{itemize}

\section{Project Outcomes}

An important objective of this project is to disseminate accessible information and tools. In addition to the mandated deliverables, there are several notable outcomes from this project. 

\subsection{Software}
\begin{itemize}
    \item {\textbf{\href{https://github.com/openpra-org/openpra-monorepo}{@openpra-org/openpra-monorepo}}: Mono repository for the OpenPRA web application~\cite{openpra_initiative_openpra_2024}}.
    
    \item {\textbf{\href{https://github.com/openpra-org/model-benchmarks}{@openpra-org/model-benchmarks}}: Automated benchmarking suite for SCRAM, XFTA, \acrshort{ftrex}, \acrshort{saphsolve} using benchexec in Docker~\cite{earthperson_pra_2025}}.

    \item {\textbf{\href{https://github.com/openpra-org/PRAcciolini}{@openpra-org/PRAcciolini}}: Automates the conversion, validation, and translation of \acrshort{pra} models. Provides interfaces for describing grammars and translation rules between schema.~\cite{earthperson_pracciolini_2024}}.
    
    \item {\textbf{\href{https://github.com/openpra-org/model-generator}{@openpra-org/model-generator}}: \acrshort{cli} utility for creating stochastically generated synthetic event trees and fault trees. Supports OpenPSA, \acrshort{saphsolve}, \acrshort{ftrex} schema. ~\cite{earthperson_pra_2022}}.
    
    \item {\textbf{\href{https://github.com/openpra-org/mef-schema}{@openpra-org/mef-schema}}: Schema definitions for OpenPRA supported model exchange formats including \acrshort{ftrex} FTP, \acrshort{saphsolve} \acrshort{jsinp}, \acrshort{jscut}, OpenPSA \acrshort{xml}s, and canopy flatbuffers~\cite{earthperson_pra_2024}}.
    
    \item {\textbf{\href{https://github.com/openpra-org/multi-hazard-model-generator}{@openpra-org/multi-hazard-model-generator}}: Generates multi-hazard event trees and fault trees in \acrshort{mar-d} from internal events \acrshort{pra} models~\cite{batikh_multi_2023}}.
    
    \item {\textbf{\href{https://github.com/a-earthperson/canopy-benchmarks}{@a-earthperson/canopy-benchmarks}}: Accuracy and performance benchmark scripts for canopy~\cite{earthperson_canopy_2025}}.
\end{itemize}

\subsection{Datasets}
\begin{itemize}
    \item \textbf{\href{https://doi.org/10.5281/zenodo.13996959}{Generic Pressurized Water Reactor (PWR) SAPHSOLVE Model}}: Reference PWR model in SAPHSOLVE format, supporting benchmarking and verification tasks~\cite{aras_generic_2024}.

    \item \textbf{\href{https://doi.org/10.5281/zenodo.14070453}{Generic \acrfull{mhtgr} Model}}~\cite{ hamza_openpra-orggeneric-mhtgr-model_2025}.
   
    \item \textbf{\href{https://doi.org/10.5281/zenodo.14070453}{Generic Pressurized Water Reactor (PWR) Open-PSA Model}}: Equivalent PWR model in OpenPSA XML for cross-tool benchmark comparisons~\cite{aras_generic_2024-1}.

    \item \textbf{\href{https://doi.org/10.5281/zenodo.13996735}{Synthetic SAPHSOLVE Models}}: Synthetically generated SAPHSOLVE format PRA models for benchmarking, quantification, and code verification~\cite{aras_synthetic_2024}.
    
    \item \textbf{\href{https://doi.org/10.5281/zenodo.13996370}{Synthetic OpenPSA Models}}: Synthetically generated OpenPSA XML format models for benchmark studies~\cite{aras_synthetic_2024-1}.
    
    \item \textbf{\href{https://doi.org/10.5281/zenodo.15320670}{openpra-org/synthetic-models}}: Centralized collection of stochastically generated PRA models in multiple supported formats, for validation and stress testing of quantification engines~\cite{aras_synthetic_2025}.
    
    \item \textbf{\href{https://doi.org/10.5281/zenodo.7706615}{Benchmarking SAPHIRE, SCRAM and XFTA}}: Dataset of synthetically generated fault trees with common cause failures, for head-to-head tool verification~\cite{earthperson_dataset_2023}.
    
    \item \textbf{\href{https://doi.org/10.5281/zenodo.15293416}{Generic OpenPSA Models: The Aralia Fault Tree Dataset}}: Large-scale, hand-curated and synthetic PRA fault trees and event trees, compatible with the OpenPSA data model~\cite{earthperson_generic_2021}.

    \item \textbf{\href{https://zenodo.org/doi/10.5281/zenodo.15320401}{Post-processing Analysis \& Supplementary Notes}}: Excelsheets, plotting analysis, MATLAB scripts used for curating and analyzing the generated raw results.~\cite{earthperson_benchmarks_2025}.
\end{itemize}

\subsection{Reports}
\begin{itemize}
    \item {Refining Processing Engines from SAPHIRE: Initialization of Fault Tree / Event Tree Solver, \acrshort{inl}, 2023~\cite{aras_refining_2023}.}
    \item {Diagnostics and Strategic Plan for Advancing the SAPHIRE Engine, \acrshort{inl}, 2023~\cite{aras_diagnostics_2023}.}
\end{itemize}
\subsubsection*{Internal Milestone Reports}
\begin{itemize}
    \item {Milestone Report 1: Literature Review, Benchmarking, \& Profiling of Current PRA Tools}, 2022~\cite{aras_milestone_2022}.
    \item {Milestone Report 2: Fine-Tuning Performance, Benchmarking, Profiling \& Design Insights}, 2023~\cite{aras_milestone_2023}.
    \item {Milestone Report 3 Task 1: Literature Review, Benchmarking, \& Profiling of Current PRA Tools}, 2024~\cite{aras_milestone_2024-1}.
    \item {Milestone Report 3 Task 2: Web-based PRA Framework Design \& Implementation}, 2024~\cite{aras_milestone_2024-2}.
    \item {Milestone Report 3 Task 3: Testing \& Benchmarking for Serial \& Parallel Improvements}, 2024~\cite{aras_milestone_2024-3}.
    \item {Milestone Report 3 Task 4: Applications to Single Hazard Real World PRA Models}, 2024~\cite{aras_milestone_2024-4}.
    \item {Milestone Report 3 Task 5: Applications to Multi-Hazard Real World PRA Models}, 2024~\cite{aras_milestone_2024-5}.
    \item {Milestone Report 3 Task 6: Verification \& Documentation}, 2024~\cite{rasheeq_milestone_2024-6}.
\end{itemize}

% \printbibliography[heading=none, keyword={my_report}]

\subsection{Journal Articles}
\begin{itemize}
    \item {A Critical Look at the Need for Performing Multi-Hazard Probabilistic Risk Assessment for Nuclear Power Plants, Eng, 2021~\cite{aras_critical_2021}.}
    \item {[Under Review] Enhancement Assessment Framework for Probabilistic Risk Assessment Tools, Reliability Engineering \& System Safety, 2025~\cite{aras_nt_2025}.}
    \item {[Under Review] A Systematic Diagnostics and Enhancement Framework for Advancing Probabilistic Risk Assessment Tools, Nuclear Technology, 2025~\cite{aras_jress_2025}.}
\end{itemize}


% \small{
% \begin{itemize}
%     \item {“A Critical Look at the Need for Performing Multi-Hazard Probabilistic Risk Assessment for Nuclear Power Plants”. In: Eng 2.4 (2021), pp. 454–467. \texttt{ISSN: 2673-4117}. \texttt{\href{https://doi.org/10.3390/eng2040028}{DOI: 10.3390/eng2040028}}}~\cite{aras_critical_2021}.
% \end{itemize}
% }

\subsection{Conference Papers}
\subsubsection*{2022}
\begin{itemize}
    % 2022
    \item {Benchmark Study of XFTA and SCRAM Fault Tree Solvers Using Synthetically Generated Fault Trees Models, \acrfull{asme} \acrfull{imece}~\cite{aras_benchmark_2022}}.
\end{itemize}

\subsubsection*{2023}
\begin{itemize}
    % 2023
    \item {Introducing OpenPRA: A Web-Based Framework for Collaborative Probabilistic Risk Assessment}, \acrshort{asme} \acrshort{imece}~\cite{earthperson_introducing_2023}.
    \item {Model Exchange Methodology Between Probabilistic Risk Assessment Tools: SAPHIRE and CAFTA Case Study}, \acrfull{ans} \acrfull{psa}~\cite{hamza_model_2023}.
    \item {Preliminary Benchmarking of \acrshort{saphsolve}, XFTA, and SCRAM Using Synthetically Generated Fault Trees with Common Cause Failures}, \acrshort{ans} \acrshort{psa}~\cite{farag_preliminary_2023}.
    \item {Methodology and Demonstration for Performance Analysis of a Probabilistic Risk Assessment Quantification Engine: SCRAM}, \acrshort{ans} \acrshort{psa}~\cite{aras_methodology_2023}.
    \item {Method of Developing a SCRAM Parallel Engine for Efficient Quantification of Probabilistic Risk Assessment Models}, \acrshort{ans} \acrshort{psa}~\cite{aras_method_2023}.
\end{itemize}

\subsubsection*{2024}
\begin{itemize}
    % 2024
    \item {Advancing SAPHIRE: Transitioning from Legacy to State-of-Art Excellence}, \acrshort{ans} \acrfull{ars}~\cite{wood_advancing_2024}.
    \item {Evaluating PRA Tools for Accurate and Efficient Quantifications: A Follow-Up Benchmarking Study Including FTREX}, \acrshort{ans} \acrshort{ars}~\cite{farag_evaluating_2024}.
    \item {Towards a Deep-Learning based Heuristic for Optimal Variable Ordering in Binary Decision Diagrams to Support Fault Tree Analysis}, \acrshort{ans} \acrshort{ars}~\cite{earthperson_towards_2024}.
    \item {Enhancing the SAPHIRE Solve Engine: Initial Progress and Efforts}, \acrshort{ans} \acrshort{ars}~\cite{aras_enhancing_2024}.
    \item {Introducing OpenPRA's Quantification Engine: Exploring Capabilities, Recognizing Limitations, and Charting the Path to Enhancement}, \acrshort{ans} \acrshort{ars}~\cite{aras_introducing_2024}.
\end{itemize}

\subsubsection*{2025 (Accepted)}
\begin{itemize}
    % 2025
    \item {Automated OpenPSA Model Generation from Reliability Diagrams Using Agentic Retrieval Augmented Generation: A Case Study on \acrshort{mhtgr}}, \acrshort{ans} \acrshort{psa}~\cite{rasheeq_automated_2025}.
    \item {Design and Implementation of a Distributed Queueing System for OpenPRA}, \acrshort{ans} \acrshort{psa}~\cite{rasheeq_design_2025}.
    \item {Synthetical Model Generator for Probabilistic Risk Assessment Tools: Enhancing Testing, Verifying and Learning}, \acrshort{ans} \acrshort{psa}~\cite{aras_synthetical_2025}.
    \item {Facilitating PRA Model Accessibility: Model Converter Utility from SAPHIRE to Open-PSA}, \acrshort{ans} \acrshort{psa}~\cite{aras_facilitating_2025}.
    \item {Probability Estimation using Monte Carlo Simulation of Boolean Logic on Hardware-Accelerated Platforms}, \acrshort{ans} \acrshort{psa}~\cite{earthperson_probability_2025}.
\end{itemize}

\subsection{Theses \& Dissertations}
\begin{itemize}
    \item {Asmaa Salem Amin Aly Farag, \textit{Benchmarking Study of Probabilistic Risk Assessment Tools Using Synthetically Generated Fault Tree Models: SAPHSOLVE, XFTA, and SCRAM}, Master of Science, Department of Nuclear Engineering, \acrshort{ncsu}, 2023~\cite{farag_thesis_2023}.}
    \item {Egemen Mutlu Aras, \textit{Enhancement Methodology for Probabilistic Risk Assessment Tools through Diagnostics, Optimization, and Parallel Computing}, Doctor of Philosophy, Department of Nuclear Engineering, \acrshort{ncsu}, 2024~\cite{aras_dissertation_2024}.}
    \item {Arjun Earthperson, \textit{[Working Title] A Data-Parallel Monte Carlo Framework for Large-Scale PRA using Probabilistic Circuits}, Doctor of Philosophy, Department of Nuclear Engineering, \acrshort{ncsu}, Expected 2025~\cite{earthperson_dissertation_2025}.}
    \item {Hasibul Hossain Rasheeq, \textit{[Working Title] Design and Implementation of a Distributed Queueing System for PRA Quantification}, Master of Science, Department of Nuclear Engineering, \acrshort{ncsu}, Expected 2025~\cite{rasheeq_thesis_2025}.}
\end{itemize}

\subsection{\color{blue}{Datasets}}

\printbibliography[heading=none, keyword={my_dataset}]


\section{\color{blue}{Recreating this Study}}
sources for this report

\section{\color{blue}{Document Layout}}
This report is organized to provide a comprehensive and logical progression from foundational concepts to the technical realization and evaluation of the developed PRA platform. The document begins with an introduction that outlines the motivation, scope, objectives, and intended outcomes of the project, as well as the specific aims of this report.

The first major part, \textit{Foundations}, establishes the theoretical and methodological basis for probabilistic risk assessment. It covers the triplet definition of risk, formalizes event and fault tree structures, and introduces the unified probabilistic directed acyclic graph (PDAG) model. This section also discusses equivalent model representations, qualitative analysis techniques, and methods for quantifying probabilities and frequencies, including both exact and approximate approaches.

The second part, \textit{Identifying Gaps}, reviews the current state of PRA software, highlighting persistent limitations in scalability, model development, and risk communication. It details the development of benchmark models, including both generic and synthetic examples, and describes the process of translating and quantifying these models using existing tools. This part also presents benchmarking results and discusses the application of the platform to multi-hazard PRA models.

The third part, \textit{Proposed High-Throughput Compute Solution}, describes the architectural design and implementation of the new platform. It covers the overall system architecture, the design of the distributed queuing system, and the development of a data-parallel Monte Carlo probability estimator. This section also addresses performance evaluation, scalability, and integration with the web-based OpenPRA platform.

The final part, \textit{Future Work}, outlines ongoing and future research directions. It identifies current limitations, such as the need for enhanced collaboration, version control, solver integration, and advanced uncertainty quantification. The report concludes with a discussion of the steps required to advance the platform toward full industry readiness and broader community adoption.

Throughout the document, figures, tables, and equations are used to clarify technical concepts and present results. References to relevant literature and standards are provided to support the discussion and facilitate further exploration by the reader.
